\chapter{Basic Probability Models}
\label{probcalc}

This chapter will introduce the general notions of probability.  Most of
it will seem intuitive to you, but pay careful attention to the general
principles which are developed; in more complex settings intuition may
not be enough, and the tools discussed here will be very useful.

\section{ALOHA Network Example}
\label{aloha}

Throughout this book, we will be discussing both ``classical''
probability examples involving coins, cards and dice, and also examples
involving applications to computer science.  The latter will involve
diverse fields such as data mining, machine learning, computer networks,
software engineering and bioinformatics.  

In this section, an example from computer networks is presented which
will be used at a number of points in this chapter.  Probability analysis
is used extensively in the development of new, faster types of networks.

Today's Ethernet evolved from an experimental network developed at the
University of Hawaii, called ALOHA.  A number of network nodes would
occasionally try to use the same radio channel to communicate with a
central computer.  The nodes couldn't hear each other, due to the
obstruction of mountains between them.  If only one of them made an
attempt to send, it would be successful, and it would receive an
acknowledgement message in response from the central computer.  But if
more than one node were to transmit, a {\bf collision} would occur,
garbling all the messages.  The sending nodes would timeout after
waiting for an acknowledgement which never came, and try sending again
later.  To avoid having too many collisions, nodes would engage in
random {\bf backoff}, meaning that they would refrain from sending for a
while even though they had something to send.

One variation is {\bf slotted} ALOHA, which divides time into intervals
which I will call ``epochs.''  Each epoch will have duration 1.0, so
epoch 1 extends from time 0.0 to 1.0, epoch 2 extends from 1.0 to 2.0
and so on.  In the version we will consider here, in each epoch, if a
node is {\bf active}, i.e. has a message to send, it will either send or
refrain from sending, with probability p and 1-p.  The value of p is set
by the designer of the network.  (Real Ethernet hardware does something
like this, using a random number generator inside the chip.) 

The other parameter q in our model is the probability that a node which
had been inactive generates a message during an epoch, i.e. the
probability that the user hits a key, and thus becomes ``active.''
Think of what happens when you are at a computer.  You are not typing
constantly, and when you are not typing, the time until you hit a key
again will be random.  Our parameter q models that randomness.

Let n be the number of nodes, which we'll assume for simplicity is two.
Assume also for simplicity that the timing is as follows.  Arrival of a
new message happens in the middle of an epoch, and the decision as to
whether to send versus back off is made near the end of an epoch, say
90\% into the epoch.

For example, say that at the beginning of the epoch which extends from
time 15.0 to 16.0, node A has something to send but node B does not.  At
time 15.5, node B will either generate a message to send or not, with
probability q and 1-q, respectively.  Suppose B does generate a new
message.  At time 15.9, node A will either try to send or refrain, with
probability p and 1-p, and node B will do the same.  Suppose A refrains
but B sends.  Then B's transmission will be successful, and at the start
of epoch 16 B will be inactive, while node A will still be active.  On
the other hand, suppose both A and B try to send at time 15.9; both will
fail, and thus both will be active at time 16.0, and so on.

Be sure to keep in mind that in our simple model here, during the time a
node is active, it won't generate any additional new messages.

(Note:  The definition of this ALOHA model is summarized concisely on
page \pageref{basicprobcomp}.)

Let's observe the network for two epochs, epoch 1 and epoch 2.  Assume
that the network consists of just two nodes, called node 1 and node 2,
both of which start out active.  Let $X_1$ and $X_2$ denote the numbers
of active nodes at the {\it very end} of epochs 1 and 2, {\it after possible
transmissions}.  We'll take p to be 0.4 and q to be 0.8 in this example.

Let's find $P(X_1 = 2)$, the probability that $X_1 = 2$, and then get to
the main point, which is to ask what we really mean by this probability.

How could $X_1 = 2$ occur?  There are two possibilities:

\begin{itemize}

\item both nodes try to send; this has probability $p^2$

\item neither node tries to send; this has probability $(1-p)^2$

\end{itemize}

Thus

\begin{equation}
\label{x12}
P(X_1 = 2) = p^2 + (1-p)^2 = 0.52
\end{equation}

\section{A ``Notebook'' View: the Notion of a Repeatable Experiment}
\label{repeatexpt}

It's crucial to understand what that 0.52 figure really means in a
practical sense.  To this end, let's put the ALOHA example aside for a
moment, and consider the ``experiment'' consisting of rolling two dice,
say a blue one and a yellow one.  Let X and Y denote the number of dots
we get on the blue and yellow dice, respectively, and consider the
meaning of $P(X+Y = 6) = \frac{5}{36}$.

\begin{table}
\begin{center}
\vskip 0.5in

\begin{tabular}{|c|c|c|c|c|c|}
\hline
1,1 & 1,2 & 1,3 & 1,4 & 1,5 & 1,6 \\
\hline
2,1 & 2,2 & 2,3 & 2,4 & 2,5 & 2,6 \\
\hline
3,1 & 3,2 & 3,3 & 3,4 & 3,5 & 3,6 \\
\hline
4,1 & 4,2 & 4,3 & 4,4 & 4,5 & 4,6 \\
\hline
5,1 & 5,2 & 5,3 & 5,4 & 5,5 & 5,6 \\
\hline
6,1 & 6,2 & 6,3 & 6,4 & 6,5 & 6,6 \\
\hline
\end{tabular}

\end{center}
\caption{Sample Space for the Dice Example}
\label{dicesamplespace}
\end{table}

In the mathematical theory of probability, we talk of a {\bf sample
space}, which (in simple cases) consists of the possible outcomes
$(X,Y)$, seen in Table \ref{dicesamplespace}.  In a theoretical
treatment, we place weights of 1/36 on each of the points in the space,
reflecting the fact that each of the 36 points is equally likely, and
then say, ``What we mean by $P(X+Y = 6) = \frac{5}{36}$ is that the
outcomes (1,5), (2,4), (3,3), (4,2), (5,1) have total weight 5/36.''

Unfortunately, the notion of sample space becomes mathematically tricky
when developed for more complex probability models.  Indeed, it requires
graduate-level math, called {\bf measure theory}.  

And much worse, under the sample space approach, one loses all the
intuition.  In particular, {\bf there is no good way using set theory to
convey the intuition underlying conditional probability} (to be
introduced in Section \ref{probdefs}).  The same is true for expected
value, a central topic to be introduced in Section \ref{expval}.

In any case, most probability computations do not rely on explicitly
writing down a sample space.  In this particular example it is useful
for us as a vehicle for explaining the concepts, but we will NOT use it
much.  Those who wish to get a more theoretical grounding can get a
start in Section \ref{reconcil}.

But the intuitive notion---which is FAR more important---of what $P(X+Y
= 6) = \frac{5}{36}$ means is the following.  Imagine doing the
experiment many, many times, recording the results in a large notebook:

\begin{itemize}

\item Roll the dice the first time, and write the outcome on the first
line of the notebook.

\item Roll the dice the second time, and write the outcome on the second
line of the notebook.

\item Roll the dice the third time, and write the outcome on the third
line of the notebook.

\item Roll the dice the fourth time, and write the outcome on the fourth
line of the notebook.

\item Imagine you keep doing this, thousands of times, filling
thousands of lines in the notebook.

\end{itemize} 

\begin{table}
\begin{center}
\vskip 0.5in

\begin{tabular}{|l|l|l|}
\hline
notebook line & outcome & blue+yellow = 6?  \\ \hline 
\hline
1 & blue 2, yellow 6 & No \\ \hline 
2 & blue 3, yellow 1 & No \\ \hline 
3 & blue 1, yellow 1 & No \\ \hline 
4 & blue 4, yellow 2 & Yes \\ \hline 
5 & blue 1, yellow 1 & No \\ \hline 
6 & blue 3, yellow 4 & No \\ \hline 
7 & blue 5, yellow 1 & Yes \\ \hline 
8 & blue 3, yellow 6 & No \\ \hline 
9 & blue 2, yellow 5 & No \\ \hline 
\end{tabular}

\end{center}
\caption{Notebook for the Dice Problem}
\label{dicenotebook} 
\end{table}

The first 9 lines of the notebook might look like Table
\ref{dicenotebook}.  Here 2/9 of these lines say Yes.  But after many,
many repetitions, approximately 5/36 of the lines will say Yes.  For
example, after doing the experiment 720 times, approximately
$\frac{5}{36} \times 720 = 100$ lines will say Yes.

This is what probability really is:  In what fraction of the lines does
the event of interest happen?  {\bf It sounds simple, but if you always
think about this ``lines in the notebook'' idea, probability problems
are a lot easier to solve.}  And it is the fundamental basis of computer
simulation.

\section{Our Definitions}
\label{probdefs}

These definitions are intuitive, rather than rigorous math, but 
intuition is what we need.  Keep in mind that we are making
\underline{definitions} below, not listing properties.

\begin{itemize}

\item We assume an ``experiment'' which is (at least in concept)
repeatable.  The experiment of rolling two dice is repeatable, and even
the ALOHA experiment is so.  (We simply watch the network for a long
time, collecting data on pairs of consecutive epochs in which there are
two active stations at the beginning.)  On the other hand, the
econometricians, in forecasting 2009, cannot ``repeat'' 2008.  Yet all
of the econometricians' tools assume that events in 2008 were affected
by various sorts of randomness, and we think of repeating the experiment
in a conceptual sense.

\item We imagine performing the experiment a large number of times,
recording the result of each repetition on a separate line in a
notebook.

\item We say A is an {\bf event} for this experiment if it is a possible
boolean (i.e. yes-or-no) outcome of the experiment.  In the above
example, here are some events:

\begin{itemize}

\item[*]  X+Y = 6

\item[*] X = 1

\item[*] Y = 3

\item[*] X-Y = 4

\end{itemize}

\item A {\bf random variable} is a numerical outcome of the experiment,
such as X and Y here, as well as X+Y, 2XY and even sin(XY).

\item For any event of interest A, imagine a column on A in the
notebook.  The $k^{th}$ line in the notebook, k = 1,2,3,..., will
say Yes or No, depending on whether A occurred or not during the
$k^{th}$ repetition of the experiment.  For instance, we have such a
column in our table above, for the event \{A = blue+yellow = 6\}.

\item For any event of interest A, we define P(A) to be the long-run
fraction of lines with Yes entries.

\item For any events A, B, imagine a new column in our notebook,
labeled ``A and B.''  In each line, this column will say Yes if and
only if there are Yes entries for both A and B.  P(A and B) is
then the long-run fraction of lines with Yes entries in the
new column labeled ``A and B.''\footnote{In most textbooks, what we call
``A and B'' here is written A$\cap$B, indicating the intersection of two
sets in the sample space.  But again, we do not take a sample space
point of view here.}

\item For any events A, B, imagine a new column in our notebook,
labeled ``A or B.''  In each line, this column will say Yes if and
only if at least one of the entries for A and B says Yes.\footnote{In
the sample space approach, this is written A $\cup$ B.}

\item For any events A, B, imagine a new column in our notebook, labeled
``A $|$ B'' and pronounced ``A given B.''  In each line: 

\begin{itemize}

\item[*] This new column will say ``NA'' (``not applicable'') if the B
entry is No.

\item[*] If it is a line in which the B column says Yes, then this
new column will say Yes or No, depending on whether the A column
says Yes or No.

\end{itemize}

\end{itemize}

Think of probabilities in this ``notebook'' context:

\begin{itemize}

\item P(A) means the long-run fraction of lines in the notebook in
which the A column says Yes.

\item P(A or B) means the long-run fraction of lines in the notebook in
which the A-or-B column says Yes.

\item P(A and B) means the long-run fraction of lines in the notebook in
which the A-and-B column says Yes.

\item P(A $|$ B) means the long-run fraction of lines in the notebook in
which the A $|$ B column says Yes---{\bf \Large among the lines which do
NOT say NA.}

\end{itemize}

{\bf A hugely common mistake is to confuse P(A and B) and P(A $|$ B).}
This is where the notebook view becomes so important.  In the dice
example, compare the quantities $P(X = 1 {\rm ~and~} S = 6) =
\frac{1}{36}$ and $P(X = 1 | S = 6) = \frac{1}{5}$, where S =
X+Y:\footnote{Think of adding an S column to the notebook too}

\begin{itemize}

\item After a large number of repetitions of the experiment,
approximately 1/36 of the lines of the notebook will have the property
that both X = 1 and S = 6 (since X = 1 and S = 6 is equivalent to X = 1
and Y = 5).

\item After a large number of repetitions of the experiment, if {\bf we
look only at the lines in which S = 6}, then {\bf among those lines},
approximately 1/5 of {\bf those lines} will show X = 1.

\end{itemize}

The quantity P(A$|$B) is called the {\bf conditional probability of A,
given B}.

Note that {\it and} has higher logical precedence than {\it or}.  For
example, P(A and B or C) means P[(A and B) or C].  Also, {\it not} has
higher precedence than {\it and}.

Here are some more very important definitions and properties:

\begin{itemize}

\item 
\begin{definition}
Suppose A and B are events such that it is impossible for them to
occur in the same line of the notebook.  They are said to be {\bf
disjoint} events.  
\end{definition}

\item If A and B are disjoint events, then 

\begin{equation}
\label{disjointor}
P(A \textrm{ or } B) = P(A) + P(B)
\end{equation}

Again, this terminology {\it disjoint} stems from the set-theoretic
sample space approach, where it means that A $\cap$ B = $\phi$.  That
mathematical terminology works fine for our dice example, but in my
experience people have major difficulty applying it correctly in more
complicated problems.  This is another illustration of why I put so much
emphasis on the ``notebook'' framework.

By writing 

\begin{equation}
\{ A \textrm{ or } B \textrm{ or } C \} =
\{ A \textrm{ or } [B \textrm{ or } C ] \} =
\end{equation}

(\ref{disjointor}) can be iterated, e.g.

\begin{equation}
P(A \textrm{ or } B \textrm{ or } C) = P(A) + P(B) + P(C)
\end{equation}

\item If A and B are not disjoint, then

\begin{equation}
\label{genor}
P(A \textrm{ or } B) = P(A) + P(B) - P(A \textrm{ and } B)
\end{equation}

In the disjoint case, that subtracted term is 0, so (\ref{genor})
reduces to (\ref{disjointor}).

\item 
\begin{definition}
Events A and B are said to be {\bf stochastically independent},
usually just stated as {\bf independent},\footnote{The term {\it
stochastic} is just a fancy synonym for {\it random}.} if 
\end{definition}

\begin{equation}
\label{indepand}
P(A \textrm{ and } B) = P(A) \cdot P(B)
\end{equation}

\item In calculating an ``and'' probability, how does one know whether
the events are independent?  The answer is that this will typically be
clear from the problem.  If we toss the blue and yellow dice, for
instance, it is clear that one die has no impact on the other, so events
involving the blue die are independent of events involving the yellow
die.  On the other hand, in the ALOHA example, it's clear that events
involving $X_1$ are NOT independent of those involving $X_2$.

\item If A and B are not independent, the equation (\ref{indepand})
generalizes to

\begin{equation}
\label{genand}
P(A \textrm{ and } B) = P(A) P(B|A)
\end{equation}

This should make sense to you.  Suppose 30\% of all UC Davis students
are in engineering, and 20\% of all engineering majors are female.  That
would imply that 0.30 x 0.20 = 0.06, i.e. 6\% of all UCD students are
female engineers.

Note that if A and B actually are independent, then $P(B|A) = P(B)$, and
(\ref{genand}) reduces to (\ref{indepand}).

Note too that (\ref{genand}) implies

\begin{equation}
\label{condba}
P(B|A) =
\frac
{ P(A \textrm{ and } B) }
{ P(A) }
\end{equation}

\end{itemize}

\section{``Mailing Tubes''}
\label{mailingtubes1}

{\it If I ever need to buy some mailing tubes, I can come here}---friend
of the author's, while browsing through an office supplies store

\bigskip

Examples of the above properties, e.g. (\ref{indepand}) and
(\ref{genand}), will be given starting in Section \ref{basicprobcomp}.
But first, a crucial strategic point in learning probability must be
addressed.

Some years ago, a friend of mine was in an office supplies store, and he
noticed a rack of mailing tubes.  My friend made the remark shown above.
Well, (\ref{indepand}) and \ref{genand} are ``mailing tubes''--make a
mental note to yourself saying, ``If I ever need to find a probability
involving {\it and}, one thing I can try is (\ref{indepand}) and
(\ref{genand}).''  {\bf \LARGE Be ready for this!}

This mailing tube metaphor will be mentioned often, such as in Section
\ref{mailingtubes2} .

\section{Example:  ALOHA Network}
\label{basicprobcomp}

Please keep in mind that the notebook idea is simply a vehicle to
help you understand what the concepts really mean.  This is crucial for
your intuition and your ability to apply this material in the real
world.    But the notebook idea is NOT for the purpose of calculating
probabilities.  Instead, we use the properties of probability, as seen
in the following. 

Let's look at all of this in the ALOHA context.  Here's a summary:

\begin{itemize}

\item We have n network nodes, sharing a common communications channel.

\item Time is divided in epochs.  $X_k$ denotes the number of active nodes at
the end of epoch k, which we will sometimes refer to as the {\bf state}
of the system in epoch k.

\item If two or more nodes try to send in an epoch, they collide, and
the message doesn't get through.

\item We say a node is active if it has a message to send.

\item If a node is active node near the end of an epoch, it tries to
send with probability p.  

\item If a node is inactive at the beginning of an epoch, then at the
middle of the epoch it will generate a message to send with probability
q.

\item In our examples here, we have n = 2 and $X_0 = 2$, i.e. both nodes
start out active.

\end{itemize}

Let's say the two nodes consist of terminals, at which John and Mary are
typing.  Call the nodes Node John and Node Mary.

Now, in Equation (\ref{x12}) we found that 

\begin{equation}
P(X_1 = 2) = p^2 + (1-p)^2 = 0.52
\end{equation}

How did we get this?  Let $C_i$ denote the event that node i
tries to send, i = 1,2.  Then using the definitions above, our steps
would be

\begin{eqnarray}
P(X_1 = 2) 
&=& P(\underbrace{ C_1 {\rm ~and~} C_2 } {\rm ~or~ } 
\underbrace{ {\rm ~not~} C_1 {\rm ~and~} ~{\rm not}~ C_2 } ) \label{a1} \\ 
&=& P(C_1 {\rm ~and~} C_2) + P(~{\rm not}~ C_1 {\rm ~and~} ~{\rm not}~ C_2) 
\textrm{ (from (\ref{disjointor}))} \label{a2} \\ 
&=& P(C_1) P(C_2) + P(~{\rm not}~ C_1) P(~{\rm not}~ C_2) 
\textrm{ (from (\ref{indepand}))} \label{a3} \\ 
&=& p^2 + (1-p)^2
\end{eqnarray}

(The underbraces in (\ref{a1}) do not represent some esoteric
mathematical operation.  There are there simply to make the grouping
clearer, corresponding to events G and H defined below.)

Here are the reasons for these steps:

\begin{itemize}

\item[(\ref{a1}):]  We listed the ways in which the event $\{X_1 = 2\}$
could occur.  

\item[(\ref{a2}):]  Write $G = {C_1 {\rm ~and~} C_2}$, $H = {D_1 {\rm
~and~} D_2}$, where $D_i = \textrm{not } C_i$, i = 1,2.  Then the events G
and H are clearly disjoint; if in a given line of our notebook there is
a Yes for G, then definitely there will be a No for H, and vice versa.

\item[(\ref{a3}):]  The two nodes act physically independently of each
other.  Thus the events $C_1$ and $C_2$ are stochastically independent,
so we applied (\ref{indepand}).  Then we did the same for $D_1$ and
$D_2$.

\end{itemize}

Now, what about $P(X_2 = 2)$?  Again, we break big events down into
small events, in this case according to the value of $X_1$:

\begin{eqnarray}
\label{x2}
P(X_2 = 2) 
&=& P(X_1 = 0 {\rm ~and~} X_2 = 2 \textrm{ or }
X_1 = 1 {\rm ~and~} X_2 = 2 \textrm{ or }
X_1 = 2 {\rm ~and~} X_2 = 2) \nonumber \\
&=& P(X_1 = 0 {\rm ~and~} X_2 = 2)  \\
&+& P(X_1 = 1 {\rm ~and~} X_2 = 2) \nonumber \\
&+& P(X_1 = 2 {\rm ~and~} X_2 = 2) \nonumber 
\end{eqnarray}

Since $X_1$ cannot be 0, that first term, $P(X_1 = 0 {\rm ~and~} X_2 =
2)$ is 0.  To deal with the second term, $P(X_1 = 1 {\rm ~and~} X_2 =
2)$, we'll use (\ref{genand}).  Due to the time-sequential nature of our
experiment here, it is natural (but certainly not ``mandated,'' as we'll
often see situations to the contrary) to take A and B to be $\{X_1 =
1\}$ and $\{X_2 = 2\}$, respectively.  So, we write

\begin{equation}
\label{x1x2}
P(X_1 = 1 {\rm ~and~} X_2 = 2) = P(X_1 = 1) P(X_2 = 2 | X_1 = 1)
\end{equation}

To calculate $P(X_1 = 1)$, we use the same kind of reasoning as in
Equation (\ref{x12}).  For the event in question to occur, either Node
John would send and Node B wouldn't, or John would refrain from sending
and Node B would send.  Thus

\begin{equation}
\label{px11}
P(X_1 = 1) = 2p(1-p) = 0.48
\end{equation}

Now we need to find $P(X_2 = 2 | X_1 = 1)$.  This again involves
breaking big events down into small ones.  If $X_1 = 1$, then $X_2 = 2$
can occur only if {\it both} of the following occur:

\begin{itemize}

\item Event A:  Whichever node was the one to successfully transmit
during epoch 1---and we are given that there indeed was one, since $X_1
= 1$---now generates a new message.

\item Event B:  During epoch 2, no successful transmission occurs, i.e.
either they both try to send or neither tries to send.

\end{itemize}

Recalling the definitions of p and q in Section \ref{aloha},  
we have that

\begin{equation}
P(X_2 = 2 | X_1 = 1) = q[p^2+(1-p)^2] = 0.41
\end{equation}

Thus $P(X_1 = 1 {\rm ~and~} X_2 = 2) = 0.48 \times 0.41 = 0.20$.

We go through a similar analysis for $P(X_1 = 2 {\rm ~and~} X_2 = 2)$:
We recall that $P(X_1 = 2) = 0.52$ from before, and find that $P(X_2 = 2
| X_1 = 2) = 0.52$ as well.  So we find $P(X_1 = 2 {\rm ~and~} X_2 = 2)$
to be $0.52^2 = 0.27$.  Putting all this together, we find that $P(X_2 =
2) = 0.47$.

Let's do another; let's find $P(X_1 = 1 | X_2 = 2)$.  [Pause a minute
here to make sure you understand that this is quite different from
$P(X_2 = 2 | X_1 = 1)$.] From (\ref{condba}), we know that 

\begin{equation}
\label{x1givx2}
P(X_1 = 1 | X_2 = 2) = \frac
{P(X_1 = 1 \textrm{ and } X_2 = 2)}
{P(X_2 = 2)}
\end{equation}

We computed both numerator and denominator here before, in Equations
(\ref{x1x2}) and (\ref{x2}), so we see that $P(X_1 = 1 | X_2 = 2) =
0.20/0.47 = 0.43$.

So, in our notebook view, if we were to look only at lines in the
notebook for which $X_2 = 2$, a fraction 0.43 of {\it those lines} 
would have $X_1 = 1$.

You might be bothered that we are looking ``backwards in time'' in
(\ref{x1givx2}), kind of guessing the past from the present.  There is
nothing wrong or unnatural about that.  Jurors in court trials do it all
the time, though presumably not with formal probability calculation.
And evolutionary biologists do use formal probability models to guess
the past.

And one more calculation:  $P(X_1 = 2 \textrm{ or } X_2 = 2)$.  From
(\ref{genor}), 

\begin{equation}
P(X_1 = 2 \textrm{ or } X_2 = 2) =
P(X_1 = 2) +
P(X_2 = 2) -
P(X_1 = 2 \textrm{ and } X_2 = 2) 
\end{equation}

Luckily, we've already calculated all three probabilities on the
right-hand side to be 0.52, 0.47 and 0.27, respectively.  Thus
$P(X_1 = 2 \textrm{ or } X_2 = 2) = 0.72$. 

Note by the way that events involving $X_2$ are NOT independent of those
involving $X_1$.  For instance, we found in (\ref{x1givx2}) that

\begin{equation}
P(X_1 = 1 | X_2 = 2) = 0.43
\end{equation}

yet from (\ref{px11}) we have

\begin{equation}
P(X_1 = 1) = 0.48.
\end{equation}

\section{Example: Dice and Conditional Probability}

Note in particular that $P(B | A)$ and $P(A | B)$ are completely
different quantities.  The first restricts attention to lines of the
notebook in which A occurs, while for the second we look at lines in
which B occurs.

Suppose two dice are rolled, resulting in the random variables X and Y.
Let S be the sum X+Y, and let T denote the number of dice having an
even number of dots, 0, 1 or 2.  Suppose we hear that S = 12, and we know
nothing else.  Then we know that T must be 2.  In other words,

\begin{equation}
P(T = 2 ~|~ S = 12) = 1
\end{equation}

On the other hand, if we hear instead that T = 2, that does {\it not}
imply that S is 12, i.e.

\begin{equation}
P(S = 12 ~|~ T = 2) < 1
\end{equation}

In other words

\begin{equation}
P(S = 12 ~|~ T = 2) \neq
P(T = 2 ~|~ S = 12)
\end{equation}

(In fact the reader should show that $P(S = 12 ~|~ T = 2) = 1/9$.

\section{ALOHA in the Notebook Context}

Think of doing the ALOHA ``experiment'' many, many times.

\begin{itemize}

\item Run the network for two epochs, starting with both nodes active,
the first time, and write the outcome on the first
line of the notebook.

\item Run the network for two epochs, starting with both nodes active,
the second time, and write the outcome on the second line of the
notebook.

\item Run the network for two epochs, starting with both nodes active,
the third time, and write the outcome on the third line of the notebook.

\item Run the network for two epochs, starting with both nodes active,
the fourth time, and write the outcome on the fourth line of the
notebook.

\item Imagine you keep doing this, thousands of times, filling
thousands of lines in the notebook.

\end{itemize} 

The first seven lines of the notebook might look like Table
\ref{alohanotebook}.  We see that:

\begin{table}
\begin{center}
\vskip 0.5in

\begin{tabular}{|l|l|l|l|l|}
\hline
notebook line & 
$X_1 = 2$ & 
$X_2 = 2$ & 
$X_1 = 2$ and $X_2 = 2$ & 
$X_2 = 2 | X_1 = 2$  \\ \hline 
\hline
1 & Yes & No & No & No \\ \hline 
2 & No & No & No & NA \\ \hline 
3 & Yes & Yes & Yes & Yes \\ \hline 
4 & Yes & No & No & No \\ \hline 
5 & Yes & Yes & Yes & Yes \\ \hline 
6 & No & No & No & NA \\ \hline 
7 & No & Yes & No & NA \\ \hline 
\end{tabular}

\end{center}
\caption{Top of Notebook for Two-Epoch ALOHA Experiment}
\label{alohanotebook}
\end{table}

\begin{itemize}

\item Among those first seven lines in the notebook, 4/7 of them have
$X_1 = 2$.  After many, many lines, this fraction will be
approximately 0.52.

\item Among those first seven lines in the notebook, 3/7 of them have
$X_2 = 2$.  After many, many lines, this fraction will be
approximately 0.47.\footnote{Don't make anything of the fact that these
probabilities nearly add up to 1.} 

\item Among those first seven lines in the notebook, 2/7 of them have
$X_1 = 2 {\rm ~and~ } X_2 = 2$.  After many, many lines, this fraction
will be approximately 0.27. 

\item Among the first seven lines in the notebook, four of them do not
say NA in the $X_2 = 2 | X_1 = 2$ column.  {\bf Among these four lines},
two say Yes, a fraction of 2/4.  After many, many lines, this
fraction will be approximately 0.52.

\end{itemize}

\section{A Note on Modeling}

Here is a question to test your understanding of the ALOHA model---not
the calculation of probabilities, but rather the meaning of the model
itself.  What kinds of properties are we trying to capture in the model?

So, consider this question:

\begin{quote} 

Consider the ALOHA network model.  Say we have two such networks, A and
B.  In A, the network typically is used for keyboard input, such as a
user typing e-mail or editing a file.  But in B, users tend to do a lot
of file uploading, not just typing.  Fill in the blanks:  In B, the
model parameter \_\_\_\_\_\_\_ is \_\_\_\_\_\_\_\_\_\_\_\_\_\_\_ than in
A, and in order to accommodate this, we should set the parameter
\_\_\_\_\_\_\_\_ to be relatively \_\_\_\_\_\_\_\_\_\_ in B.

\end{quote}

In network B we have heavy traffic.  Thus, when a node becomes idle, it
is quite likely to have a new message to send right away.\footnote{The
file is likely read in chunks called disk {\it sectors}, so there may be
a slight pause between the uploading of chunks.  Our model here is too
coarse to reflect such things.}  Thus q is large.

That means we need to be especially worried about collisions, so we
probably should set p to a low value.

\section{Solution Strategies}

The example in Section \ref{basicprobcomp} shows typical stategies in
exploring solutions to probability problems, such as:

\begin{itemize}

\item Name what seem to be the important variables and events, in this case 
$X_1$, $X_2$, $C_1$, $C_2$ and so on.

\item Write the given probability in terms of those named variables,
e.g.  

\begin{equation}
P(X_1 = 2) 
= P(\underbrace{ C_1 {\rm ~and~} C_2 } {\rm ~or~ } 
\underbrace{ {\rm ~not~} C_1 {\rm ~and~} ~{\rm not}~ C_2 } ) 
\end{equation}

above.

\item Ask the famous question, ``How can it happen?''  
Break big events down into small events; in the above case
the event $X_1 = 2$ can happen if $C_1 
\textrm{ and } 
C_2 
\textrm{ or } 
\textrm{ not } 
C_1 
\textrm{ and } 
\textrm{ not } 
C_2$.

\item But when you do break things down like this, make sure to neither
expand or contract the scope of the probability.  Say you write
something like

\begin{equation}
P(A) = P(B)
\end{equation}

where B might be some complicated event expression such as in the
right-hand side of (\ref{a1}).  Make SURE that A and B are
equivalent---meaning that in every notebook line in which A occurs, then
B also occurs, and {\it vice versa}.

\item Do not write/think nonsense.  For example: the expression ``P(A)
or P(B)'' is nonsense---do you see why?  Probabilities are numbers, not
boolean expressions, so ``P(A) or P(B)'' is like saying, ``0.2 or
0.5''---meaningless!

Similarly, say we have a random variable X.  The ``probability'' P(X) is
invalid.  Say X is the number of dots we get when we roll a single die.
Then P(X) would mean ``the probability that the number of dots,'' which
is nonsense English!  P(X = 3) is valid, but P(X) is meaningless. 

Please note that = is not like a comma, or equivalent to the English
word {\it therefore}.  It needs a left side and a right side; ``a = b''
makes sense, but ``= b'' doesn't.

\item Similarly, don't use ``formulas'' that you didn't learn and that
are in fact false.  For example, in an expression involving a random
variable X, one can NOT replace X by its mean.  (How would you like it
if your professor were to lose your exam, and then tell you, ``Well,
I'll just assign you a score that is equal to the class mean''?)

\item Adhere to convention!  Use capital letters for random variables
and names of events.  Use P() notation, not p() (which will mean
something else in this book).

\item In the beginning of your learning probability methods,
meticulously write down all your steps, with reasons, as in the
computation of $P(X_1 = 2)$ in Equations (\ref{a1})ff.  After you gain
more experience, you can start skipping steps, but not in the initial
learning period.

\item Solving probability problems---and even more so, building useful
probability models---is like computer programming:  It's a creative
process.

One can NOT---repeat, NOT---teach someone how to write programs.  All
one can do is show the person how the basic building blocks work, such
as loops, if-else and arrays, then show a number of examples.  But the
actual writing of a program is a creative act, not formula-based.  The
programmer must creatively combined the various building blocks to
produce the desired result.  The teacher cannot teach the student how to
do this.

The same is true for solving probability problems.  The basic building
blocks were presented above in Section \ref{basicprobcomp}, and many
more ``mailing tubes'' will be presented in the rest of this book.  But
it is up to the student to try using the various building blocks in a
way that solves the problem.  Sometimes use of one block may prove to be
unfruitful, in which case one must try other blocks.

For instance, in using probability formulas like P(A and B) = P(A)
P(B$|$A), there is no magic rule as to how to choose A and B.  

Moreover, if you need P(B$|$A), there is no magic rule on how to find
it.  On the one hand, you might calculate it from (\ref{condba}), as we
did in (\ref{x1givx2}), but on the other hand you may be able to reason
out the value of P(B$|$A), as we did following (\ref{px11}).  Just try
some cases until you find one that works, in the sense that you can
evaluate both factors.  It's the same as trying various programming
ideas until you find one that works.

\end{itemize}

% \section{Example:  Divisibility of Random Integers}
% 
% Suppose at step i we generate a random integer between 1 and 1000, and
% check whether it's evenly divisible by i, i = 5,4,3,2,1.  Let N denote
% the number of steps needed to reach an evenly divisible number.
% 
% Let's find P(N = 2).  Let q(i) denote the fraction of numbers in
% 1...,1000 that are evenly divisible by i, so that for instance q(5) =
% 200/1000 = 1/5 while q(3) = 333/1000.  Let's label the steps 5,4,..., so
% that the first step is number 5.  Then since the random numbers are
% independent from step to step, we have
% 
% \begin{eqnarray}
% \label{divis}
% P(N = 2) &=& P(\textrm{fail in step 5 and succeed in step 4}) ~~~~
% \textrm{(``How can it happen?'')}\\
% &=& P(\textrm{fail in step 5) ~ P(succeed in step 4 $|$ fail in step 5)} ~~~~
% ((\ref{genand})) \\
% &=& [1-q(5)] q(4) \\
% &=& \frac{4}{5} \cdot \frac{1}{4} \\
% &=& \frac{1}{5}
% \end{eqnarray}
% 
% But there's more.
% 
% First, note that q(i) is either equal or approximately equal to 1/i.
% Then following the derivation in (\ref{divis}), you'll find that
% 
% \begin{equation}
% P(N = j) \approx \frac{1}{5}
% \end{equation}
% 
% for ALL j in 1,...,5.  
% 
% That may seem counterintuitive.  Yet the example here is in essence the
% same as one found as an exercise in many textbooks on probability:
% 
% \begin{quote}
% A man has five keys.  He knows one of them opens a given lock, but he
% doesn't know which.  So he tries the keys one at a time until he finds
% the right one.  Find P(N = j), j = 1,...,5, where N is the number of
% keys he tries until he succeeds. 
% \end{quote}
% 
% Here too the answer is 1/5 for all j.  But this one makes intuitive
% sense:  Each of the keys has chance 1/5 of being the right key, so each
% of the values 1,...,5 is equally likely for N.
% 
% This is then an example of the fact that sometimes we can gain insight
% into one problem by considering a mathematically equivalent problem in a
% quite different setting. 
% 
\section{Example:  A Simple Board Game} 
\label{boardgame}

Consider a board game, which for simplicity we'll assume consists of two
squares per side, on four sides.  A player's token advances around the
board.  The squares are numbered 0-7, and play begins at square 0.  The
token advances according to the roll of a single die.  If a player lands
on square 3, he/she gets a bonus turn.

In most problems like this, {\bf it is greatly helpful to first name the
quantities or events involved}.  Toward that end, let R denote the
player's first roll, and let B be his bonus if there is one, with B
being set to 0 if there is no bonus.  

Let's find the probability that a player has yet to make a complete
circuit of the board---i.e. has not yet reached or passed 0---after the
first turn (including the bonus, if any).

\begin{eqnarray}
P(\textrm{doesn't reach or pass 0}) &=& P(R+B \leq 7) \\ 
&=& P(R \leq 6, R \neq 3 \textrm{ or } R = 3, B \leq 4) \\
&=& P(R \leq 6, R \neq 3) + P(R = 3, B \leq 4) \\
&=& P(R \leq 6, R \neq 3) + P(R = 3) ~ P(B \leq 4 ~|~ R = 3) \\
&=& \frac{5}{6} + \frac{1}{6} \cdot \frac{4}{6} \\
&=& \frac{17}{18}
\end{eqnarray}

(Above we have written commas as a shorthand notation for {\it and}, a
common abbreviation.)

Now here is a very subtle issue.  The events $R+B = 3$ and $B \leq 4$
are \textit{not} independent.  At first it would seem that they're
independent, as we are rolling the die twice, and successful rolls are
independent.  That's true, they are independent, but remember, $B$ can
be 0 (no bonus), and the event $B \leq 4$ includes that 0 case.  And if
we know $R = 3$, then we know for sure that $B$ cannot be 0, so we see
the two events are not independent.

Note that we first had to ``translate'' the English specification of the
problem (``doesn't reach or...'') to math symbols.  As mentioned, this
is quite helpful.

Now, here's a shorter way (there are always multiple ways to do a
problem):

\begin{eqnarray}
P(\textrm{don't reach or pass 0}) &=& 1 - P(\textrm{do reach or pass 0}) \\ 
&=& 1 - P(R + B > 7) \\
&=& 1 - P(R = 3,  B > 4) \\
&=& 1 - \frac{1}{6} \cdot \frac{2}{6} \\
&=& \frac{17}{18}
\end{eqnarray}

Now suppose that, according to a telephone report of the game, you hear
that on the player's first turn, his token ended up at square 4.  Let's 
find the probability that he got there with the aid of a bonus roll.

{\bf Note that this a conditional probability}---we're finding the
probability that the player got a bonus roll, \underline{given} that we
know he ended up at square 4.  The word {\it given} wasn't there in the
statement of the problem, but it was implied.

A little thought reveals that we cannot end up at square 4 after making
a complete circuit of the board, which simplifies the situation quite a
bit.  So, write

\begin{eqnarray}
\label{condboard}
P(B > 0 ~|~ R+B = 4) &=& \frac{P(R+B 
= 4, B > 0)}{P(R+B=4)} \\ &=& \frac{P(R+B = 4, B > 0)}{P(R+B = 4, B > 0 \textrm{ or } R+B = 4, B = 0)} \\
&=& \frac{P(R+B = 4, B > 0)}{P(R+B = 4, B > 0) + P(R+B = 4, B = 0)} \\
&=& \frac{P(R = 3, B = 1)}{P(R = 3, B = 1) + P(R = 4)} \\
&=& \frac
{\frac{1}{6} \cdot \frac{1}{6}}
{\frac{1}{6} \cdot \frac{1}{6} + \frac{1}{6}} \\
&=& \frac{1}{7}
\end{eqnarray}

We could have used Bayes' Rule to shorten the derivation a little here,
but will prefer to derive everything, at least in this introductory
chapter.

Pay special attention to that third equality above, as it is a frequent
mode of attack in probability problems.  In considering the probability
P(R+B = 4, B $>$ 0), we ask, what is a simpler---but still
equivalent!---description of this event?  Well, we see that R+B = 4, B
$>$ 0 boils down to R = 3, B = 1, so we replace the above probability
with P(R = 3, B = 1).

Again, this is a very common approach.  But be sure to take care that we
are in an ``if and only if'' situation.  Yes, R+B = 4, B $>$ 0 implies R
= 3, B = 1, but we must make sure that the converse is true as well.  In
other words, we must also confirm that R = 3, B = 1 implies R+B = 4, B
$>$ 0.  That's trivial in this case, but one can make a subtle error in
some problems if one is not careful; otherwise we will have replaced a
higher-probability event by a lower-probability one.

\section{Example:  Bus Ridership}
\label{busridership}

Consider the following analysis of bus ridership.  (In order to keep
things easy, it will be quite oversimplified, but the principles will be
clear.)  Here is the model:

\begin{itemize}

\item At each stop, each passsenger alights from the bus, independently,
with probability 0.2 each.  

\item Either 0, 1 or 2 new passengers get on the bus, with probabilities
0.5, 0.4 and 0.1, respectively.  Passengers at successive stops are
independent.

\item Assume the bus is so large that it never becomes full, so the new
passengers can always get on. 

\item Suppose the bus is empty when it arrives at its first stop.

\end{itemize}

Once again, {\bf it is greatly helpful to first name the quantities or
events involved}.  Let $L_i$ denote the number of passengers on the bus
as it {\it leaves} its i$^{th}$ stop, i = 1,2,3,...  Let $B_i$ denote
the number of new passengers who board the bus at the i$^{th}$ stop.  

First, let's find the probability that no passengers board the bus at
the first three stops.  That's easy:

\begin{equation}
P(B_1 = 0 \textrm{ and } B_2 = 0 \textrm{ and } B_3 = 0) = 0.5^3
\end{equation}

Now let's compute the probability that the bus leaves the second stop
empty.  Again, we must translate this to math first, $P(L_2 = 0)$.  Now
as usual, ``break big events into small events'':

\begin{eqnarray}
P(L_2 = 0) &=& 
P(B_1 = 0 \textrm{ and } L_2 = 0 \textrm{ or }
B_1 = 1 \textrm{ and } L_2 = 0 \textrm{ or }
B_1 = 2 \textrm{ and } L_2 = 0) \\
&=& \sum_{i=0}^2 P(B_1 = i \textrm{ and } L_2 = 0) \\
&=& \sum_{i=0}^2 P(B_1 = i) P(L_2 = 0 | B_1 = i) \\
&=& 0.5^2 + (0.4) (0.2) (0.5) + (0.1) (0.2^2) (0.5) \\
&=& 0.292 \label{eq292}
\end{eqnarray}

For instance, where did that first term, $0.5^2$, come from?
Well, $P(B_1 = 0) = 0.5$, and what about $P(L_2 = 0 | B_1 = 0)$?
If $B_1 = 0$, then the bus approaches the second stop empty.  For it to
then {\it leave} that second stop empty, it must be the case that $B_2 =
0$, which has probability 0.5.  In other words, $P(L_2 = 0 | B_1 = 0) =
0.5$.

As another example, suppose we are told that the bus arrives empty at the
third stop.  What is the probability that exactly two people boarded the
bus at the first stop?  

Note first what is being asked for here:  $P(B_1 = 2 | L_2 = 0)$.  Then
we have

\begin{eqnarray}
P(B_1 = 2 ~|~ L_2 = 0) &=&
\frac{P(B_1 = 2 \textrm{ and } L_2 = 0)}{P(L_2 = 0)} \\
&=& P(B_1 = 2) ~ P(L_2 = 0 ~|~ B_1 = 2) ~ /~  0.292 \\
&=& 0.1 \cdot 0.2^2 \cdot 0.5 / 0.292
\end{eqnarray}

(the 0.292 had been previously calculated in (\ref{eq292})).

Now let's find the probability that fewer people board at the second
stop than at the first:

\begin{eqnarray}
P(B_2 < B_1)
&=& P(B_1 = 1 \textrm{ and } B_2 < B_1 \textrm{ or }
      B_1 = 2 \textrm{ and } B_2 < B_1) \\
&=& 0.4 \cdot 0.5 +
    0.1 \cdot (0.5 + 0.4)
\end{eqnarray}

Also:  Someone tells you that as she got off the bus at the second stop,
she saw that the bus then left that stop empty.  Let's find the
probability that she was the only passenger when the bus left the first
stop:

We are given that $L_2 = 0$.  But we are {\it also} given that
$L_1 > 0$.  Then

\begin{eqnarray}
P(L_1 = 1 | L_2 = 0 \textrm{ and } L_1 > 0)&=&
   \frac
      {P(L_1 = 1 \textrm{ and } L_2 = 0)}
      {P(L_2 = 0 \textrm{ and } L_1 > 0)} \\
&=&
   \frac{P(B_1 = 1 \textrm{ and } L_2 = 0)}
   {P(
      B_1 = 1 \textrm{ and } L_2 = 0 \textrm{ or }
      B_1 = 2 \textrm{ and } L_2 = 0)} \label{l1l2} \\
&=& \frac
      {(0.4)(0.2)(0.5)}
      {(0.4)(0.2)(0.5) + (0.1)(0.2)^2(0.5)}
\end{eqnarray}

Equation (\ref{l1l2}) requires some explanation.  Let's first consider
how we got the numerator from the preceding equation.

Ask the usual question:  How can it happen?  In this case, how can the
event 

\begin{equation}
L_1 = 1 \textrm{ and } L_2 = 0
\end{equation}

occur?  Since we know a lot about
the probabilistic behavior of the $B_i$, let's try to recast that event.
A little thought shows that the event is equivalent to the event 

\begin{equation}
B_1 = 0 \textrm{ and } L_2 = 0
\end{equation}

So, how did the denominator in (\ref{l1l2}) come from the preceding
equation?  In other words, how did we recast the event 

\begin{equation}
L_2 = 0 \textrm{ and } L_1 > 0
\end{equation}

in terms of the $B_i$?  Well, $L_1 > 0$ means that $B_1$ is either 1 or
2.  Thus we broke things down accordingly in the denominator of
(\ref{l1l2}).  

The remainder of the computation is similar to what we did earlier in
this example.

Here is another computation:  An observer at the second stop notices
that no one alights there, but it is dark and the observer couldn't see
whether anyone was on the bus. Find the probability that there was one
passenger on the bus at the time.

Let $A_i$ denote the number of passengers alighting at stop i.

\begin{eqnarray}
P(L_1 = 1 | A_2 = 0)
&=& \frac
{P(L_1 = 1 \textrm{ and } A_2 = 0)}
{P(A_2 = 0)} \\ 
&=& \frac
{P(L_1 = 1 \textrm{ and } A_2 = 0)}
{\sum_{j=0}^2
P(L_1 = j \textrm{ and } A_2 = 0)} \\
&=& 
\frac
{0.4 \cdot 0.8}
{0.5 \cdot 1 + 0.4 \cdot 0.8 + 0.1 \cdot 0.8^2}
\end{eqnarray}

\section{Bayes' Rule}
\label{bayesthm}

(This section should not be confused with Section \ref{bayesian}.  The
latter is highly controversial, while the material in this section is
not controversial at all.)

Following (\ref{x1givx2}) above, we noted that the ingredients had
already been computed, in (\ref{x1x2}) and (\ref{x2}).  If we go back
to the derivations in those two equations and substitute in
(\ref{x1givx2}), we have

\begin{eqnarray}
P(X_1 = 1 | X_2 = 2) &=& 
\frac
{P(X_1 = 1 \textrm{ and } X_2 = 2)}
{P(X_2 = 2)} \\
&=& \frac
{P(X_1 = 1 \textrm{ and } X_2 = 2)}
{P(X_1 = 1 \textrm{ and } X_2 = 2) + P(X_1 = 2 \textrm{ and } X_2 = 2)} \\
&=& \frac
{P(X_1 = 1) P(X_2 = 2 | X_1 = 1)}
{P(X_1 = 1) P(X_2 = 2 | X_1 = 1) + P(X_1 = 2) P(X_2 = 2 | X_1 = 2)} 
\end{eqnarray}

Looking at this in more generality, for events A and B we would find
that

\begin{equation}
\label{thisisbayes}
P(A|B) = \frac{P(A) P(B|A)}{P(A) P(B|A) + P(\textrm{not }A)
P(B|\textrm{not } A)}
\end{equation}

This is known as {\bf Bayes' Theorem} or {\bf Bayes' Rule}.  It can be
extended easily to cases with several terms in the denominator, arising
from situations that need to be broken down into several subevents
rather than just A and not-A.

\section{Random Graph Models}
\label{randomgraph}
A {\it graph} consists of {\it vertices} and {\it edges}.  To understand
this, think of a social network.  Here the vertices represent people and
the edges represent friendships.  For the time being, assume that
friendship relations are mutual, i.e.\ if person i says he is friends
with person j, then j will say the same about i.

For any graph, its {\it adjacency matrix} consists of 1 and 0 entries,
with a 1 in row i, column j meaning there is an edge from vertex i to
vertex j.  For instance, say we have a simple tiny network of three
people, with adjacency matrix

\begin{equation}
\label{graphex}
\left (
\begin{array}{rrr}
0 & 1 & 1 \\
1 & 0 & 0 \\
1 & 0 & 0 \\
\end{array}
\right )
\end{equation}

Row 1 of the matrix says that Person 1 is friends with persons 2 and 3,
but we see from the other rows that Persons 2 and 3 are not friends with
each other.

In any graph, the {\it degree} of a vertex is its number of edges.  So,
the degree of vertex i is the number of 1s in row i.  In the little
model above, vertex 1 has degree 2 but the other two vertices each have
degree 1.

The assumption that friendships are mutual is described in graph theory
as having a {\it undirected} graph.  Note that that implies that the
adjacency matrix is symmetric.  However, we might model some other
networks as {\it directed}, with adjacency matrices that are not
necessarily symmetric.  In a large extended family, for example, we
could define edges in terms of being an elder sibling; there would be an
edge from Person i to Person j if j is an older sibling of i.

Graphs need not represent people.  They are used in myriad other
settings, such as analysis of Web site relations, Internet traffic
routing, genetics research and so on.

\subsection{Example:  Preferential Attachment Graph Model}
\label{prefattach}

A famous graph model is {\it Preferential Attachment}.  Think of it
again as an undirected social network, with each edge representing a
``friend'' relation.  The number of vertices grows over time, one vertex
per time step.  At time 0, we have just two vertices, $v_1$ and $v_2$,
with a link between them.  

Thus at time 0, each of the two vertices has degree 1.  Whenever a new
vertex is added to the graph, it randomly chooses an existing vertex to
{\it attach to}, creating a new edge with that existing vertex.  In
making that random choice, it follows probabilities in proportion to the
degrees of the existing edges.


As an example of how the Preferential Attachment Model works, suppose
that at just before time 2, when $v_4$ is added, the adjacency matrix
for the graph is (\ref{graphex}).  Then there will be an edge created
between $v_4$ with $v_1$, $v_2$ or $v_3$, with probability 2/4, 1/4 and
1/4, respectively.  

Let's find $P(v_4 \textrm{ attaches to } v_1)$ (in general, no longer
assuming (\ref{graphex})).  Let $N_i$ denote the node that $v_i$
attaches to, i = 3,4,...  Then, following the solution strategy ``break
big event down into small events,'' let's break this question about
$v_4$ according to what happens with $v_3$: 

\begin{eqnarray}
P(N_4 = 1)
&=& P(N_3 = 1 \textrm{ and } N_4 = 1) \
+ P(N_3 = 2 \textrm{ and } N_4 = 1) \\ 
&=& (1/2) (2/4) + (1/2) (1/4) \\
&=& 3/8
\end{eqnarray}

\section{Simulation}

Computer simulation essentially does in actual code what one does
conceptually in our ``notebook'' view of probability (Section
\ref{repeatexpt}). 

\subsection{Example:  Rolling Dice}

If we roll three dice, what is the probability that their total is 8?
We could count all the possibilities, or we could get an approximate answer
via simulation:

\begin{Verbatim}[fontsize=\relsize{-2},numbers=left]
# roll d dice; find P(total = k)

probtotk <- function(d,k,nreps) {
   count <- 0
   # do the experiment nreps times -- like doing nreps notebook lines
   for (rep in 1:nreps) {
      sum <- 0
      # roll d dice and find their sum
      for (j in 1:d) sum <- sum + roll()
      if (sum == k) count <- count + 1
   }
   return(count/nreps)
}

# simulate roll of one die; the possible return values are 1,2,3,4,5,6,
# all equally likely
roll <- function() return(sample(1:6,1))

# example
probtotk(3,8,1000)
\end{Verbatim}

The call to the built-in R function {\bf sample()} here says to take a
sample of size 1 from the sequence of numbers 1,2,3,4,5,6.  That's just
what we want to simulate the rolling of a die.  The code

\begin{Verbatim}[fontsize=\relsize{-2}]
for (j in 1:d) sum <- sum + roll()
\end{Verbatim}

then simulates the tossing of a die d times, and computing the sum.

\subsection{First Improvement}
\label{improvingsimcode}

Since applications of R often use large amounts of computer time, good R
programmers are always looking for ways to speed things up.  Here is an
alternate version of the above program:

\begin{Verbatim}[fontsize=\relsize{-2},numbers=left]
# roll d dice; find P(total = k)

probtotk <- function(d,k,nreps) {
   count <- 0
   # do the experiment nreps times
   for (rep in 1:nreps) 
      total <- sum(sample(1:6,d,replace=TRUE))
      if (total == k) count <- count + 1
   }
   return(count/nreps)
}
\end{Verbatim}

Let's first discuss the code.

\begin{Verbatim}[fontsize=\relsize{-2}]
sample(1:6,d,replace=TRUE)
\end{Verbatim}

The call to {\bf sample()} here says, ``Generate d random numbers,
chosen randomly (i.e.\ with equal probability) from the integers 1
through 6, with replacement.''  Well, of course, that simulates tossing
the die d times.  So, that call returns a d-element array, and we then
call R's built-in function {\bf sum()} to find the total of the d dice.

Note the call to R's {\bf sum()} function, a nice convenience.

This second version of the code here eliminates one explicit loop, which
is the key to writing fast code in R.  But just as important, it is more
compact and clearer in expressing what we are doing in this simulation.

\subsection{Second Improvement}

Further improvements are possible.  Consider this code: 

\begin{Verbatim}[fontsize=\relsize{-2},numbers=left]
# roll d dice; find P(total = k)

# simulate roll of nd dice; the possible return values are 1,2,3,4,5,6,
# all equally likely
roll <- function(nd) return(sample(1:6,nd,replace=TRUE))

probtotk <- function(d,k,nreps) {
   sums <- vector(length=nreps)
   # do the experiment nreps times
   for (rep in 1:nreps) sums[rep] <- sum(roll(d))
   return(mean(sums==k))
}
\end{Verbatim}

There is quite a bit going on here.

We are storing the various ``notebook lines'' in a vector {\bf sums}.
We first call {\bf vector()} to allocate space for it.

But the heart of the above code is the expression {\bf sums==k}, which
involves the very essence of the R idiom, {\bf vectorization}.  At
first, the expression looks odd, in that we are comparing a vector
(remember, this is what languages like C call an {\it array}), {\bf
sums}, to a scalar, {\bf k}.  But in R, every ``scalar'' is actually
considered a one-element vector.  

Fine, {\bf k} is a vector, but wait!  It has a different length than
{\bf sums}, so how can we compare the two vectors?  Well, in R 
a vector is {\bf recycled}---extended in length, by repeating its
values---in order to conform to longer vectors it will be involved with.
For instance:

\begin{Verbatim}[fontsize=\relsize{-2}]
> c(2,5) + 4:6
[1]  6 10  8
\end{Verbatim}

Here we added the vector (2,5) to (4,5,6).  The former was first
recycled to (2,5,2), resulting in a sum of (6,10,8).\footnote{There was
also a warning message, not shown here.  The circumstances under which
warnings are or are not generated are beyond our scope here, but 
recycling is a very common R operation.}

So, in evaluating the expression {\bf sums==k}, R will recycle {\bf k}
to a vector consisting of {\bf nreps} copies of {\bf k}, thus conforming
to the length of {\bf sums}.  The result of the comparison will then be
a vector of length {\bf nreps}, consisting of TRUE and FALSE values.
In numerical contexts, these are treated at 1s and 0s, respectively.
R's {\bf mean()} function will then average those values, resulting in
the fraction of 1s!  That's exactly what we want.

\subsection{Third Improvement}
\label{thirdbetter}

Even better:

\begin{Verbatim}[fontsize=\relsize{-2},numbers=left]
roll <- function(nd) return(sample(1:6,nd,replace=TRUE))

probtotk <- function(d,k,nreps) {
   # do the experiment nreps times
   sums <- replicate(nreps,sum(roll(d)))
   return(mean(sums==k))
}
\end{Verbatim}

R's {\bf replicate()} function does what its name implies, in this case
executing the call {\bf sum(roll(d))}.  That produces a vector, which we
then assign to {\bf sums}.  And note that we don't have to allocate space
for {\bf sums}; {\bf replicate()} produces a vector, allocating space,
and then we merely point {\bf sums} to that vector.

The various improvements shown above compactify the code, and in many
cases, make it much faster.\footnote{You can measure times using R's
{\bf system.time()} function, e.g. via the call {\bf
system.time(probtotk(3,7,10000))}.}  Note, though, that this comes at
the expense of using more memory.

\subsection{Example: Dice Problem}

Suppose three fair dice are rolled.  We wish to find the approximate
probability that the first die shows fewer than 3 dots, {\it given} that the
total number of dots for the 3 dice is more than 8, using simulation.

Again, simulation is writing code the implements our ``notebook'' view of
probability.  In this case, we are working with a conditional
probability, which our notebook view defined as follows.  $P(B ~|~ A)$
is the long-run proportion of the time B occurs, {\it among those lines
in which A occurs}.  Here is the code:

\begin{lstlisting}[numbers=left]
dicesim <- function(nreps) {
   count1 <- 0
   count2 <- 0
   for (i in 1:nreps) {
      d <- sample(1:6,3,replace=T)
      if (sum(d) > 8) {  # "among those lines in which A occurs"
         count1 <- count1 + 1
         if (d[1] < 3) count2 <- count2 + 1
      }
   }
   return(count2 / count1)
}
\end{lstlisting}

Note carefully that we did NOT use (\ref{condba}).  That would defeat
the purpose of simulation, which is the model the actual process.

\subsection{Use of runif() for Simulating Events}

To simulate whether a simple event occurs or not, we typically use R
function {\bf runif()}.  This function generates random numbers from the
interval (0,1), with all the points inside being equally likely.  So for
instance the probability that the function returns a value in (0,0.5) is
0.5.  Thus here is code to simulate tossing a coin:

\begin{Verbatim}[fontsize=\relsize{-2}]
if (runif(1) < 0.5) heads <- TRUE else heads <- FALSE
\end{Verbatim}

The argument 1 means we wish to generate just one random number from the
interval (0,1).

\subsection{Example: ALOHA Network (cont'd.)}
\label{alohasim}

Following is a computation via simulation of the {\it approximate}
values of $P(X_1 = 2)$, $P(X_2 = 2)$ and $P(X_2 = 2 | X_1 = 1)$. 


\begin{Verbatim}[fontsize=\relsize{-2},numbers=left]
# finds P(X1 = 2), P(X2 = 2) and P(X2 = 2|X1 = 1) in ALOHA example
sim <- function(p,q,nreps) {
   countx2eq2 <- 0
   countx1eq1 <- 0
   countx1eq2 <- 0
   countx2eq2givx1eq1 <- 0
   # simulate nreps repetitions of the experiment
   for (i in 1:nreps) {
      numsend <- 0  # no messages sent so far
      # simulate A and B's decision on whether to send in epoch 1
      for (j in 1:2)
         if (runif(1) < p) numsend <- numsend + 1
      if (numsend == 1)  X1 <- 1
      else X1 <- 2
      if (X1 == 2) countx1eq2 <- countx1eq2 + 1
      # now simulate epoch 2
      # if X1 = 1 then one node may generate a new message
      numactive <- X1
      if (X1 == 1 && runif(1) < q) numactive <- numactive + 1
      # send?
      if (numactive == 1) 
         if (runif(1) < p) X2 <- 0
         else X2 <- 1
      else {  # numactive = 2
         numsend <- 0
         for (i in 1:2)
            if (runif(1) < p) numsend <- numsend + 1
         if (numsend == 1) X2 <- 1
         else X2 <- 2
      }
      if (X2 == 2) countx2eq2 <- countx2eq2 + 1
      if (X1 == 1) {  # do tally for the cond. prob.
         countx1eq1 <- countx1eq1 + 1
         if (X2 == 2) countx2eq2givx1eq1 <- countx2eq2givx1eq1 + 1
      }
   }
   # print results
   cat("P(X1 = 2):",countx1eq2/nreps,"\n")
   cat("P(X2 = 2):",countx2eq2/nreps,"\n")
   cat("P(X2 = 2 | X1 = 1):",countx2eq2givx1eq1/countx1eq1,"\n")
}
\end{Verbatim}

Note that each of the {\bf nreps} iterations of the main {\bf for} loop
is analogous to one line in our hypothetical notebook.  So, to find
(the approximate value of) $P(X_1 = 2)$, divide the count of the number
of times $X_1 = 2$ occurred by the number of iterations.

Again, note especially that the way we calculated $P(X_2 = 2 | X_1 = 1)$
was to count the number of times $X_2 = 2$, {\bf among those times that}
$X_1 = 1$, just like in the notebook case.

Also:  Keep in mind that we did NOT use (\ref{thisisbayes}) or any other
formula in our simulation.  We stuck to basics, the ``notebook''
definition of probability.  This is really important if you are using
simulation to confirm something you derived mathematically.  On the
other hand, if you are using simulation because you CAN'T derive
something mathematically (the usual situation), using some of the
mailing tubes might speed up the computation.

\subsection{Example:  Bus Ridership (cont'd.)}
\label{busridershipsim}

Consider the example in Section \ref{busridership}.  Let's find the
probability that after visiting the tenth stop, the bus is empty.  This
is too complicated to solve analytically, but can easily be simulated:

\begin{Verbatim}[fontsize=\relsize{-2},numbers=left]
nreps <- 10000
nstops <- 10
count <- 0
for (i in 1:nreps) {
   passengers <- 0
   for (j in 1:nstops) {
      if (passengers > 0) 
         for (k in 1:passengers) 
            if (runif(1) < 0.2) 
               passengers <- passengers - 1
      newpass <- sample(0:2,1,prob=c(0.5,0.4,0.1))
      passengers <- passengers + newpass
   }
   if (passengers == 0) count <- count + 1
}
print(count/nreps)
\end{Verbatim}

Note the different usage of the {\bf sample()} function in the call

\begin{Verbatim}[fontsize=\relsize{-2}]
sample(0:2,1,prob=c(0.5,0.4,0.1))
\end{Verbatim}

Here we take a sample of size 1 from the set \{0,1,2\}, but with
probabilities 0.5 and so on.  Since the third argument for {\bf
sample()} is {\bf replace}, not {\bf prob}, we need to specify the
latter in our call.

\subsection{Example: Board Game (cont'd.)}

Recall the board game in Section \ref{boardgame}.  Below is simulation
code to find the probability in (\ref{condboard}):

\begin{Verbatim}[fontsize=\relsize{-2},numbers=left]
boardsim <- function(nreps) {
   count4 <- 0
   countbonusgiven4 <- 0
   for (i in 1:nreps) {
      position <- sample(1:6,1)
      if (position == 3) {
         bonus <- TRUE
         position <- (position + sample(1:6,1)) %% 8
      } else bonus <- FALSE
      if (position == 4) {
         count4 <- count4 + 1
         if (bonus) countbonusgiven4 <- countbonusgiven4 + 1
      }
   }
   return(countbonusgiven4/count4)
}
\end{Verbatim}

\subsection{Example:  Broken Rod}
\label{brokenrod}

Say a glass rod drops and breaks into 5 random pieces.  Let's find the
probability that the smallest piece has length below 0.02.  

First, what does ``random'' mean here?  Let's assume that the break
points, treating the left end as 0 and the right end as 1, can be
modeled with {\bf runif()}.  Here then is code to do the job:

\begin{lstlisting}
# random breaks the rod in k pieces, returning the length of the
# shortest one
minpiece <- function(k) {
   breakpts <- sort(runif(k-1))
   lengths <- diff(c(0,breakpts,1))
   min(lengths)
}

# returns the approximate probability that the smallest of k pieces will
# have length less than q
bkrod <- function(nreps,k,q) {
   minpieces <- replicate(nreps,minpiece(k))
   mean(minpieces < q) 
}

> bkrod(10000,5,0.02) 
[1] 0.35
\end{lstlisting}

So, we generate the break points according to the model, then sort them
in order to call R's {\bf diff()} function.  The latter finds
differences between successvie values of its argument, which in our case
will give us the lengths of the pieces.  We then find the minimum
length.

\subsection{Example: Toss a Coin Until $k$ Consecutive Heads}

We toss a coin until we get $k$ heads in a row.  Let $N$ denote the
number of tosses needed, so that for instance the pattern HTHHH gives $N
= 5$ for $k = 3$.  Here is code that finds the approximate probability
that $N > m$:

\begin{lstlisting}
ngtm <- function(k,m,nreps) {
   count <- 0
   for (rep in 1:nreps) {
      consech <- 0
      for (i in 1:m) {
         toss <- sample(0:1,1)
         if (toss) {
            consech <- consech + 1
            if (consech == k) break
         } else consech <- 0
      }
      if (consech < k) count <- count + 1
   }
   return(count/nreps)
}
\end{lstlisting}

\subsection{How Long Should We Run the Simulation?}

Clearly, the large the value of {\bf nreps} in our examples above, the
more accurate our simulation results are likely to be.  But how large
should this value be?  Or, more to the point, what measure is there for
the degree of accuracy one can expect (whatever that means) for a given
value of {\bf nreps}?  These questions will be addressed in Chapter
\ref{chap:confints}.

\section{Combinatorics-Based Probability Computation}
\label{comb}

{\it And though the holes were rather small, they had to count them
all}---from the Beatles song, {\it A Day in the Life}

\bigskip

In some probability problems all the outcomes are equally likely.  The
probability computation is then simply a matter of counting all the
outcomes of interest and dividing by the total number of possible
outcomes.  Of course, sometimes even such counting can be challenging,
but it is simple in principle.  We'll discuss two examples here.

\subsection{Which Is More Likely in Five Cards, One King or Two
Hearts?}
\label{hearts}

Suppose we deal a 5-card hand from a regular 52-card deck.  Which is
larger, P(1 king) or P(2 hearts)?  Before continuing, take a moment to
guess which one is more likely.

Now, here is how we can compute the probabilities.  {\bf The key point is
that all possible hands are equally likely, which implies that all we
need to do is count them.}  There are $\binom{52}{5}$ possible hands, so
this is our denominator.  For P(1 king), our numerator will be the
number of hands consisting of one king and four non-kings.  Since there
are four kings in the deck, the number of ways to choose one king is
$\binom{4}{1} = 4$.  There are 48 non-kings in the deck, so there are
$\binom{48}{4}$ ways to choose them.  Every choice of one king can be
combined with every choice of four non-kings, so the number of hands
consisting of one king and four non-kings is $4 \cdot \binom{48}{4}$.
Thus

\begin{equation}
P(1 \textrm{~ king}) = \frac{4 \cdot \binom{48}{4}}{\binom{52}{5}} = 0.299
\end{equation}

The same reasoning gives us

\begin{equation}
P(2 \textrm{~ hearts}) = \frac{\binom{13}{2} \cdot \binom{39}{3}}{\binom{52}{5}} = 0.274
\end{equation}

So, the 1-king hand is just slightly more likely.

Note that an unstated assumption here was that all 5-card hands are
equally likely.  That {\it is} a realistic assumption, but it's
important to understand that it plays a key role here.

By the way, I used the R function {\bf choose()} to evaluate these
quantities, running R in interactive mode, e.g.:

\begin{Verbatim}[fontsize=\relsize{-2}]
> choose(13,2) * choose(39,3) / choose(52,5)
[1] 0.2742797
\end{Verbatim}

R also has a very nice function {\bf combn()} which will generate all
the $\binom{n}{k}$ combinations of k things chosen from n, and also will
at your option call a user-specified function on each combination.  This
allows you to save a lot of computational work.  See the examples in R's
online documentation.

Here's how we could do the 1-king problem via simulation:

\begin{Verbatim}[fontsize=\relsize{-2},numbers=left]
# use simulation to find P(1 king) when deal a 5-card hand from a
# standard deck

# think of the 52 cards as being labeled 1-52, with the 4 kings having
# numbers 1-4

sim <- function(nreps) {
   count1king <- 0  # count of number of hands with 1 king
   for (rep in 1:nreps) {
      hand <- sample(1:52,5,replace=FALSE)  # deal hand
      kings <- intersect(1:4,hand)  # find which kings, if any, are in hand
      if (length(kings) == 1) count1king <- count1king + 1
   }
   print(count1king/nreps)
}
\end{Verbatim}

Here the {\bf intersect()} function performs set intersection, in this
case the set {1,2,3,4} and the one in the variable {\bf hand}.  Applying
the {\tt length()} function then gets us number of kings.

\subsection{Example:  Random Groups of Students}

A class has 68 students, 48 of which are CS majors.  The
68 students will be randomly assigned to groups of 4.  Find the
probability that a random group of 4 has exactly 2 CS majors.

$$
\frac
{\binom{48}{2} ~ \binom{20}{2}}
{\binom{68}{4}}
$$


\subsection{Example:  Lottery Tickets}
\label{lottery}

Twenty tickets are sold in a lottery, numbered 1 to 20,
inclusive.  Five tickets are drawn for prizes.  Let's find the
probability that two of the five winning tickets are
even-numbered.

Since there are 10 even-numbered tickets, there are $\binom{10}{2}$ sets
of two such tickets.  Continuing along these lines, we find the desired
probability to be.

\begin{equation}
\frac
{\binom{10}{2} \binom{10}{3}}
{\binom{20}{5}}
\end{equation}

Now let's find the probability that two of the five winning tickets are
in the range 1 to 5, two are in 6 to 10, and one is in 11 to 20.  

Picture yourself picking your tickets.  Again there are $\binom{20}{5}$
ways to choose the five tickets.  How many of those ways satisfy the
stated condition?

Well, first, there are $\binom{5}{2}$ ways to choose two tickets from
the range 1 to 5.  Once you've done that, there are $\binom{5}{2}$ ways
to choose two tickets from the range 6 to 10, and so on.  So, The
desired probability is then

\begin{equation}
\frac
{\binom{5}{2} \binom{5}{2} \binom{10}{1}} 
{\binom{20}{5}}
\end{equation}

\subsection{``Association Rules'' in Data Mining}
\label{assocrules}

The field of {\it data mining} is a branch of computer science, but it
is largely an application of various statistical methods to really huge
databases.

One of the applications of data mining is called the {\it market basket}
problem.  Here the data consists of records of sales transactions, say
of books at Amazon.com.  The business' goal is exemplified by Amazon's
suggestion to customers that ``Patrons who bought this book also tended
to buy the following books.''\footnote{Some customers appreciate such tips,
while others view it as insulting or an invasion of privacy, but we'll
not address such issues here.}  The goal of the market basket problem is
to sift through sales transaction records to produce {\it association
rules}, patterns in which sales of some combinations of books imply
likely sales of other related books.

The notation for association rules is $A,B \Rightarrow C,D,E$, meaning
in the book sales example that customers who bought books A and B also
tended to buy books C, D and E.  Here A and B are called the {\bf
antecedents} of the rule, and C, D and E are called the {\bf
consequents}.  Let's suppose here that we are only interested in rules
with a single consequent.  

We will present some methods for finding good rules in another chapter, but
for now, let's look at how many possible rules there are.  Obviously, it
would be impractical to use rules with a large number of
antecedents.\footnote{In addition, there are serious statistical
problems that would arise, to be discussed in another chapter.}.  Suppose
the business has a total of 20 products available for sale.  What
percentage of potential rules have three or fewer
antecedents?\footnote{Be sure to note that this is also a probability,
namely the probability that a randomly chosen rule will have three or
fewer antecedents.}

For each k = 1,...,19, there are $\binom{20}{k}$ possible sets of k
antecedents, and for each such set there are $\binom{20-k}{1}$ possible
consequents.  The fraction of potential rules using three or fewer
antecedents is then

\begin{equation}
\frac
{\sum_{k=1}^{3} \binom{20}{k} \cdot \binom{20-k}{1}}
{\sum_{k=1}^{19} \binom{20}{k} \cdot \binom{20-k}{1}}
= \frac{23180}{10485740}
= 0.0022
\end{equation}

So, this is just scratching the surface.  And note that with only 20
products, there are already over ten million possible rules.  With 50
products, this number is $2.81 \times 10^{16}$!  Imagine what happens in
a case like Amazon, with millions of products.  These staggering numbers
show what a tremendous challenge data miners face.




\subsection{Multinomial Coefficients}
\label{multnomcoeff}

Question:  We have a group consisting of 6 Democrats, 5 Republicans and
2 Independents, who will participate in a panel discussion.  They will
be sitting at a long table.  How many seating arrangements are possible,
with regard to political affiliation?  (So we do not care, for instance,
about permuting the individual Democrats within the seats assigned to
Democrats.)

Well, there are $\binom{13}{6}$ ways to choose the Democratic seats.
Once those are chosen, there are $\binom{7}{5}$ ways to choose the
Republican seats.  The Independent seats are then already determined,
i.e. there will be only way at that point, but let's write it as
$\binom{2}{2}$.  Thus the total number of seating arrangements is

\begin{equation}
\frac{13!}{6!7!} \cdot
\frac{7!}{5!2!} \cdot
\frac{2!}{2!0!} 
\end{equation}

That reduces to

\begin{equation}
\frac{13!}{6!5!2!}
\end{equation}

The same reasoning yields the following:

{\bf Multinomial Coefficients:} Suppose we have c objects and r bins.
Then the number of ways to choose $c_1$ of them to put in bin 1, $c_2$
of them to put in bin 2,..., and $c_r$ of them to put in bin r is

\begin{equation}
\frac{c!}{c_1!...c_r!}, ~~~ c_1+...+c_r = c
\end{equation}

Of course, the ``bins'' may just be metaphorical.  In the political
party example above, the ``bins '' were political parties, and
``objects'' were seats.

\subsection{Example:  Probability of Getting Four Aces in a Bridge
Hand}

A standard deck of 52 cards is dealt to four players, 13 cards each.
One of the players is Millie.  What is the probability that Millie is
dealt all four aces?

Well, there are

\begin{equation}
\frac{52!}{13!13!13!13!}
\end{equation}

possible deals.  (the ``objects'' are the 52 cards, and the ``bins'' are
the 4 players.) The number of deals in which Millie holds all four aces
is the same as the number of deals of 48 cards, 9 of which go to Millie
and 13 each to the other three players, i.e.  

\begin{equation}
\frac{48!}{13!13!13!9!}
\end{equation}

Thus the desired probability is

\begin{equation}
\frac
{
\frac{48!}{13!13!13!9!}
}
{
\frac{52!}{13!13!13!13!}
}
= 0.00264
\end{equation}

\section{Computational Complements}

\subsection{More on the replicate() Function}

The call form of {\bf replicate()} is

\begin{lstlisting}
replicate(numberOfReplications,codeBlock
\end{lstlisting}

In our example in Section \ref{thirdbetter},

\begin{lstlisting}
sums <- replicate(nreps,sum(roll(d)))
\end{lstlisting}

{\bf codeBlock} was just a single statement, a call to R's {\bf sum()}
function.  If more than one statement is to be executed, it must be done
so in a {\it block}, a set of statements enclosed by braces, such as

\begin{lstlisting}
f <- function() 
{
   replicate(3,
      {
         x <- sample(1:10,5,replace=TRUE)
         c(mean(x),sd(x))
      }
    )
}
\end{lstlisting}

