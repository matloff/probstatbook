\chapter{The Normal Distributions}
\label{chap:normal}

Again, these are the famous ``bell-shaped curves,'' so called because
their densities have that shape.

\section{Density and Properties}

The density for a normal distribution is

\begin{equation}
\label{nointegral}
f_W(t) = \frac{1}{\sqrt{2\pi} \sigma} ~ e^{- 0.5 \left (\frac{t-\mu}{\sigma}
\right )^2}, -\infty < t < \infty
\end{equation}

Again, this is a two-parameter family, indexed by the parameters $\mu$
and $\sigma$, which turn out to be the mean\footnote{Remember, this is a
synonym for expected value.} and standard deviation  $\mu$ and $\sigma$,
The notation for it is $N(\mu,\sigma^2)$ (it is customary to state
the variance $\sigma^2$ rather than the standard deviation).

And we write 

\begin{equation}
X ~ \widetilde{ } ~ N(\mu,\sigma^2)
\end{equation}

to mean that the random variable X has the distribution $N(\mu,\sigma^2)$.
(The tilde is read ``is distributed as.'')


{\bf Note:} Saying ``X has a $N(\mu,\sigma^2)$ distribution'' is {\it
more} than simply saying ``X has mean $\mu$ and variance $\sigma^2$.''
The former statement tells us not only the mean and variance of X, but
{\it also} the fact that X has a ``bell-shaped'' density in the
(\ref{nointegral}) family.

\subsection{Closure Under Affine Transformation}
\label{affine}

The family is closed under affine transformations:

\begin{quote}

If 

\begin{equation}
X ~ \widetilde{} ~ N(\mu,\sigma^2)
\end{equation}

and we set 

\begin{equation}
Y = cX + d 
\end{equation}

then

\begin{equation}
Y ~ \widetilde{} ~ N(c\mu+d,c^2\sigma^2)
\end{equation}

\end{quote}

For instance, suppose X is the height of a randomly selection UC Davis
student, measured in inches.  Human heights do have approximate normal
distributions; a histogram plot of the student heights would look
bell-shaped.  Now let Y be the student's height in centimeters.  Then we
have the situation above, with c = 2.54 and d = 0.  The claim about
affine transformations of normally distributed random variables would
imply that a histogram of Y would again be bell-shaped.

Consider the above statement carefully.  

\begin{quote}

It is saying much more than simply that Y has mean $c\mu + d$ and
variance $c^2\sigma^2$, which would follow from our our ``mailing
tubes'' such as (\ref{varcu}) {\it even if X did not have a normal
distribution}.  The key point is that this new variable Y is also a
member of the normal family, i.e.\ its density is still given by
(\ref{nointegral}), now with the new mean and variance.

\end{quote}

Let's derive this, using the reasoning of Section \ref{densgofx}.

For convenience, suppose $c > 0$.  Then

\begin{eqnarray}
F_Y(t) &=& P(Y \leq t)  ~~ (\textrm{definition of } F_Y)  \\ 
&=& P(cX + d \leq t) ~~ (\textrm{definition of Y}) \\
&=& P \left ( X \leq \frac{t-d}{c} \right ) ~~ (\textrm{algebra}) \\
&=& F_X \left (\frac{t-d}{c} \right )~~ (\textrm{definition of } F_X)
\label{xcdf}
\end{eqnarray}

Therefore

\begin{eqnarray}
f_Y(t) &=& \frac{d}{dt} F_Y(t) ~~ (\textrm{definition of } f_Y) \\
&=& \frac{d}{dt} F_X \left (\frac{t-d}{c} \right ) ~~ (\textrm{from } 
(\ref{xcdf})) \\      
&=& f_X \left ( \frac{t-d}{c} \right ) \cdot \frac{d}{dt} \frac{t-d}{c}
~~ (\textrm{definition of } f_X \textrm{ and the Chain Rule}) \\
&=& \frac{1}{c} \cdot 
\frac{1}{\sqrt{2\pi} \sigma} ~ e^{- 0.5 
\left (\frac{\frac{t-d}{c}-\mu}{\sigma} \right )^2} 
~~ (\textrm{from } (\ref{nointegral}) \\
&=& \frac{1}{\sqrt{2\pi} (c \sigma)} ~ 
e^{- 0.5 \left (\frac{t-(c\mu +d)}{c\sigma} \right )^2} ~~ (\textrm{algebra})
\end{eqnarray}

That last expression is the $N(c\mu+d,c^2\sigma^2)$ density, so we are
done!

\subsection{Closure Under Independent Summation}
\label{sumindep}

If X and Y are independent random variables, each having a normal
distribution, then their sum S = X + Y also is normally distributed.

This is a pretty remarkable phenomenon, not true for most other
parametric families.  If for instance X and Y each with, say, a U(0,1)
distribution, then the density of S turns out to be triangle-shaped, NOT
another uniform distribution.  (This can be derived using the methods of
Section \ref{convolution}.)

Note that if X and Y are independent and normally distributed, then the
two properties above imply that cX + dY will also have a normal
distribution, for any constants c and d.

More generally:

\begin{quote}
For constants $a_1,...,a_k$ and {\it independent} random variables
$X_1,...,X_k$, with

\begin{equation}
X_i ~ \widetilde{ } ~ N(\mu_i, \sigma_i^2
\end{equation}

form the new random variable $Y = a_1 X_1 +...+ a_k X_k$.   Then

\begin{equation}
\label{lincombnormal}
Y ~ \widetilde{ } ~ N(\sum_{i=1}^k a_i \mu_i, \sum_{i=1}^k a_i^2 \sigma_i^2)
\end{equation}
\end{quote}

{\bf Lack of intuition:}

The reader should ponder how remarkable this property of the normal
family is, because there is no intuitive explanation for it.

Imagine random variables $X$ and $Y$, each with a normal distribution.
Say the mean and variances are 10 and 4 for $X$, and 18 and 6 for $Y$.
We repeat our experiment 1000 times for our ``notebook,'' i.e.\ 1000
lines with 2 columns.  If we draw a histogram of the $X$ column, we'll
get a bell-shaped curve, and the same will be true for the $Y$ column.

But now add a $Z$ column, for $Z = X + Y$.  Why in the world should a
histogram of the $Z$ column also be bell-shaped?

\section{R Functions}

\begin{lstlisting}
dnorm(x, mean = 0, sd = 1)
pnorm(q, mean = 0, sd = 1)
qnorm(p, mean = 0, sd = 1)
rnorm(n, mean = 0, sd = 1)
\end{lstlisting}

Here {\bf mean} and {\bf sd} are of course the mean and standard
deviation of the distribution.  The other arguments are as in our
previous examples.

\section{The Standard Normal Distribution}
\label{stdnorm}

\begin{definition}

If $Z ~ \widetilde{} ~ N(0,1)$ we say the random variable Z has a {\it
standard normal distribution}.

Note that if $X ~ \widetilde{} ~ N(\mu,\sigma^2)$, and if we set

\begin{equation}
Z = \frac{X - \mu}{\sigma}
\end{equation}

then 

\begin{equation}
\label{zisn01}
Z ~ \widetilde{} ~ N(0,1)
\end{equation}

\end{definition}

The above statements follow from the earlier material:

\begin{itemize}

\item Define $Z = \frac{X - \mu}{\sigma}$.

\item Rewrite it as $Z = \frac{1}{\sigma} \cdot X +
(\frac{-\mu}{\sigma})$.

% \item By (\ref{aubv}) we know that
\item Since E(cU + d) = c EU + d for any random variable U and constants
c and d, we have

\begin{equation}
EZ = \frac{1}{\sigma} EX - \frac{\mu}{\sigma} = 0
\end{equation}

and (\ref{affinevar}) and (\ref{varcu}) imply that Var(X) = 1.

\item OK, so we know that Z has mean 0 and variance 1.  But does it have
a normal distribution?  Yes, due to our discussion above titled
``Closure Under Affine Transformations."

\end{itemize}

By the way, the N(0,1) cdf is traditionally denoted by $\Phi$.  

\section{Evaluating Normal cdfs}

The function in (\ref{nointegral}) does not have a closed-form
indefinite integral.  Thus probabilities involving normal random
variables must be approximated.  Traditionally, this is done with a
table for the cdf of N(0,1), which is included as an appendix to almost
any statistics textbook; the table gives the cdf values for that
distribution.

But this raises a question:  There are infinitely many distributions in
the normal family.  Don't we need a separate table for each?  That of
course would not be possible, and in fact it turns out that this one
table---the one for the N(0,1) distribution--- is sufficient for the
entire normal family.  Though we of course will use R to get such
probabilities, it will be quite instructive to see how these table
operations work.

Here's why one table is enough: Say X has an $N(10,2.5^2)$ distribution.
How can we get a probability like, say, $P(X < 12)$ using the N(0,1)
table?  Write

\begin{equation}
P(X < 12) = P\left (Z < \frac{12-10}{2.5}\right ) = P(Z < 0.8)
\end{equation}

Since on the right-hand side Z has a standard normal distribution,
we can find that latter probably from the N(0,1) table!



As noted, traditionally it has played a central role, as one could
transform any probability involving some normal distribution to an
equivalent probability involving N(0,1).  One would then use a table of
N(0,1) to find the desired probability.

The transformation $Z = (X - \mu)/\sigma$ will play a big role in other
contexts in future chapters, but for the sole purpose of simply
evaluating normal probabilities, we can be much more direct.  Nowadays,
probabilities for any normal distribution, not just N(0,1), are easily
available by computer.  In the R statistical package, the normal cdf for
any mean and variance is available via the function {\bf pnorm()}.  The
call form is

\begin{Verbatim}[fontsize=\relsize{-2}]
pnorm(q,mean=0,sd=1)
\end{Verbatim}

This returns the value of the cdf evaluated at {\bf q}, for a normal
distribution having the specified mean and standard deviation (default
values of 0 and 1).

We can use {\bf rnorm()} to simulate normally distributed random
variables.  The call is 

\begin{Verbatim}[fontsize=\relsize{-2}]
rnorm(n,mean=0,sd=1)
\end{Verbatim}

which returns a vector of {\bf n} random variates from the specified
normal distribution.

There are also of course the corresponding density and quantile
functions, {\bf dnorm()} and {\bf qnorm()}.

\section{Example:  Network Intrusion}
\label{netintrude}

As an example, let's look at a simple version of the network intrusion
problem.  Suppose we have found that in Jill's remote logins to a
certain computer, the number X of disk sectors she reads or writes has
an approximate normal distribution with a mean of 500 and a standard
deviation of 15.

Before we continue, a comment on modeling: Since the number of sectors
is discrete, it could not have an exact normal distribution. But then,
no random variable in practice has an exact normal or other continuous
distribution, as discussed in Section \ref{unicorns}, and the
distribution can indeed by approximately normal.

Now, say our network intrusion monitor finds that Jill---or someone
posing as her---has logged in and has read or written 535 sectors.
Should we be suspicious?  

To answer this question, let's find $P(X \geq 535)$:  Let $Z =
(X-500)/15$.  From our discussion above, we know that Z has a N(0,1)
distribution, so

\begin{equation}
P(X \geq 535) = P \left (Z \geq \frac{535-500}{15} \right ) 
= 1 - \Phi(35/15) = 0.01
\end{equation}

Again, traditionally we would obtain that 0.01 value from a N(0,1) cdf
table in a book.  With R, we would just use the function {\bf pnorm()}:

\begin{Verbatim}[fontsize=\relsize{-2}]
> 1 - pnorm(535,500,15)
[1] 0.009815329
\end{Verbatim}

Anyway, that 0.01 probability makes us suspicious.  While it {\it could}
really be Jill, this would be unusual behavior for Jill, so we start to
suspect that it isn't her.  It's suspicious enough for us to probe more
deeply, e.g. by looking at which files she (or the impostor)
accessed---were they rare for Jill too?

Now suppose there are two logins to Jill's account, accessing X and Y
sectors, with X+Y = 1088.  Is this rare for her, i.e. is $P(X+Y >
1088)$? small? 

We'll assume X and Y are independent.  We'd have to give some thought as
to whether this assumption is reasonable, depending on the details of
how we observed the logins, etc., but let's move ahead on this basis.

From page \pageref{sumindep}, we know that the sum S = X+Y is again normally
distributed.  Due to the properties in Chapter \ref{dis}, we know S has
mean $2 \cdot 500$ and variance $2 \cdot 15^2$.  The desired
probability is then found via

\begin{lstlisting}
1 - pnorm(1088,1000,sqrt(450))
\end{lstlisting}

which is about 0.00002.  That is indeed a small number, and we should be
highly suspicious.

Note again that the normal model (or any other continuous model) can
only be approximate, especially in the tails of the distribution, in
this case the right-hand tail.  But it is clear that S is only rarely
larger than 1088, and the matter mandates further investigation.

Of course, this is very crude analysis, and real intrusion detection
systems are much more complex, but you can see the main ideas here.

\section{Example:  Class Enrollment Size}
\label{classize}

After years of experience with a certain course, a university has found
that online pre-enrollment in the course is approximately normally
distributed, with mean 28.8 and standard deviation 3.1.  Suppose that in
some particular offering, pre-enrollment was capped at 25, and it hit
the cap.  Find the probability that the actual demand for the course was
at least 30.

Note that this is a conditional probability!  Evaulate it as follows.
Let N be the actual demand.  Then the key point is that we are given that
$N \geq 25$, so

\begin{eqnarray}
P(N \geq 30 | N \geq 25) 
&=&  
\frac
{P(N \geq 30 \textrm{ and } N \geq 25)}
{P(N \geq 25)} ~~~~ ((\ref{genand})) \\
&=&  
\frac
{P(N \geq 30)}
{P(N \geq 25)} \\
&=& 
\frac
{1 -\Phi \left [ (30-28.8)/3.1 \right ] }
{1 -\Phi \left [ (25-28.8)/3.1 \right ] } \\
&=& 0.39
\end{eqnarray}

Sounds like it may be worth moving the class to a larger room before
school starts.

Since we are approximating a discrete random variable by a continuous
one, it might be more accurate here to use a {\bf correction for
continuity}, described in Section \ref{correctcontin}.

\section{More on the Jill Example}

Continuing the Jill example, suppose there is never an intrusion, i.e.
all logins are from Jill herself.  Say we've set our network intrusion
monitor to notify us every time Jill logs in and accesses 535 or more
disk sectors.  In what proportion of all such notifications will Jill
have accessed at least 545 sectors?  

This is $P(X \geq 545 $|$ X \geq 535)$.  By an analysis similar
to that in Section \ref{classize}, this probability is

\begin{lstlisting}
(1 - pnorm(545,500,15)) / (1 - pnorm(535,500,15))
\end{lstlisting}

\section{Example:  River Levels}

Consider a certain river, and L, its level (in feet) relative to its
average. There is a flood whenever L $>$ 8, and it is reported that 
2.5\% of days have flooding.  Let's assume that the level 
L is normally distributed; the above information implies that the mean
is 0. 

Suppose the standard deviation of L, $\sigma$, goes up by 10\%. How much
will the percentage of flooding days increase?

To solve this, let's first find $\sigma$.  We have that

\begin{equation}
0.025 = P(L > 8) = P \left ( \frac{L-0}{\sigma} > \frac{8-0}{\sigma}
\right )
\end{equation}

Since $(L-0)/\sigma$ has a N(0,1) distribution, we can find the 0.975
point in its cdf:

\begin{lstlisting}
> qnorm(0.975,0,1)
[1] 1.959964
\end{lstlisting}

So,

\begin{equation}
1.96 =  \frac{8-0}{\sigma}
\end{equation}

so $\sigma$ is about 4.

If it increases to 4.4, then we can evaluate $P(L > 8)$ by

\begin{lstlisting}
> 1 - pnorm(8,0,4.4)
[1] 0.03451817
\end{lstlisting}

So, a 10\% increase in $\sigma$ would lead in this case to about a 40\%
increase in flood days.

\section{Example:  Upper Tail of a Light Bulb Distribution}

Suppose we model light bulb lifetimes as having a normal
distribution with mean and standard deviation 500 and 50 hours,
respectively.  Give a loop-free R expression for finding the value of d
such that 30\% of all bulbs have lifetime more than d.

You should develop the ability to recognize when we need {\bf p}-series
and {\bf q}-series functions.  Here we need

\begin{lstlisting}
qnorm(1-0.30,500,50)
\end{lstlisting}

\section{The Central Limit Theorem}
\label{theclt}

The Central Limit Theorem (CLT) says, roughly speaking, that a random
variable which is a sum of many components will have an approximate
normal distribution.  So, for instance, human weights are approximately
normally distributed, since a person is made of many components.  The
same is true for SAT test scores,\footnote{This refers to the raw
scores, before scaling by the testing company.} as the total score is
the sum of scores on the individual problems.

There are many versions of the CLT.  The basic one requires that the
summands be independent and identically distributed:\footnote{A more
mathematically precise statement of the theorem is given in Section
\ref{formalclt}.}

\begin{theorem}
\label{impreciseclt}

Suppose $X_1, X_2, ...$ are independent random variables, all having the
same distribution which has mean m and variance $v^2$.  Form the new
random variable $T = X_1+...+X_n$.  Then for large n, the distribution
of T is approximately normal with mean nm and variance $nv^2$.

\end{theorem}

The larger n is, the better the approximation, but typically n = 20 or
even n = 10 is enough.

\section{Example:  Cumulative Roundoff Error}

Suppose that computer roundoff error in computing the square roots of
numbers in a certain range is distributed uniformly on (-0.5,0.5), and
that we will be computing the sum of n such square roots.  Suppose we
compute a sum of 50 square roots.  Let's find the approximate
probability that the sum is more than 2.0 higher than it should be.
(Assume that the error in the summing operation is negligible compared
to that of the square root operation.)

Let $U_1,...,U_{50}$ denote the errors on the individual terms in the
sum.  Since we are computing a sum, the errors are added too, so our
total error is

\begin{equation}
T = U_1 + ... + U_{50}
\end{equation}

By the Central Limit Theorem, since T is a sum, it has an approximately
normal distribution, with mean 50 EU and variance 50 Var(U), where U is
a random variable having the distribution of the $U_i$.  From Section
\ref{unifprops}, we know that 

\begin{equation}
EU = (-0.5+0.5) / 2 = 0, ~~ Var(U) = \frac{1}{12} [0.5-(-0.5)]^2 =
\frac{1}{12}
\end{equation}

So, the approximate distribution of T is N(0,50/12).  We can then use R to
find our desired probability:

\begin{lstlisting}
> 1 - pnorm(2,mean=0,sd=sqrt(50/12))
[1] 0.1635934
\end{lstlisting}

\section{Example:  R Evaluation of a Central Limit Theorem
Approximation}

Say $W = U_1 + ... + U_{50}$, with the $U_i$ being independent and
identically distributed (i.i.d.) with uniform distributions on (0,1).
Give an R expression for the approximate value of $P(W < 23.4)$.

W has an approximate normal distribution, with mean $50 \times
0.5$ and variance $50 \times (1/12)$.  So we need

\begin{lstlisting}
pnorm(23.4,25,sqrt(50/12))
\end{lstlisting}

\section{Example:  Bug Counts}

As an example, suppose the number of bugs per 1,000 lines of code has a
Poisson distribution with mean 5.2.  Let's find the probability of
having more than 106 bugs in 20 sections of code, each 1,000 lines long.
We'll assume the different sections act independently in terms of bugs.

Here $X_i$ is the number of bugs in the i$^{th}$ section of code, and T
is the total number of bugs.  This is another clear candidate for using
the CLT.

Since each $X_i$ has a Poisson distribution, $m = v^2 = 5.2$.  So, T,
being a sum, is approximately distributed normally with mean and
variance $20 \times 5.2$.  So, we can find the approximate probability
of having more than 106 bugs:

\begin{Verbatim}[fontsize=\relsize{-2}]
> 1 - pnorm(106,20*5.2,sqrt(20*5.2))
[1] 0.4222596
\end{Verbatim}

\section{Example:  Coin Tosses}
\label{correctcontin}

Binomially distributed random variables, though discrete, also are
approximately normally distributed.  Here's why:

Say T has a binomial distribution with n trials.  Then we  
can write T as a sum of indicator random variables (Section
\ref{indicator}):

\begin{equation}
T = T_1+...+T_n
\end{equation}

where $T_i$ is 1 for a success and 0 for a failure on the i$^{th}$
trial.  Since we have a sum of independent, identically distributed
terms, the CLT applies.  Thus we use the CLT if we have binomial
distributions with large n.

For example, let's find the approximate probability of getting more than
12 heads in 20 tosses of a coin.  X, the number of heads, has a binomial
distribution with n = 20 and p = 0.5  Its mean and variance are then
np = 10 and np(1-p) = 5.  So, let $Z = (X-10)/\sqrt{5}$, and write

\begin{equation}
\label{gt12}
P(X > 12) = P(Z > \frac{12-10}{\sqrt{5}}) 
\approx 1 - \Phi(0.894) = 0.186
\end{equation}

Or:

\begin{Verbatim}[fontsize=\relsize{-2}]
> 1 - pnorm(12,10,sqrt(5))
[1] 0.1855467
\end{Verbatim}

The exact answer is 0.132, not too close.  Why such a big error?
The main reason is n here is rather small.  But actually, we can still
improve the approximation quite a bit, as follows.

Remember, the reason we did the above normal calculation was that X is
approximately normal, from the CLT.  This is an approximation of the
distribution of a discrete random variable by a continuous one, which
introduces additional error.

We can get better accuracy by using the {\bf correction of continuity},
which can be motivated as follows.  As an alternative to (\ref{gt12}),
we might write

\begin{equation}
P(X > 12) = P( X \geq 13) = P(Z > \frac{13-10}{\sqrt{5}}) 
\approx 1 - \Phi(1.342) = 0.090
\end{equation}

That value of 0.090 is considerably smaller than the 0.186 we got from
(\ref{gt12}).  We could ``split the difference'' this way:

\begin{equation}
P(X > 12) = P( X \geq 12.5) = P(Z > \frac{12.5-10}{\sqrt{5}}) 
\approx 1 - \Phi(1.118) = 0.132
\end{equation}

(Think of the number 13 ``owning'' the region between 12.5 and 13.5, 14
owning the part between 13.5 and 14.5 and so on.) Since the exact answer
to seven decimal places is 0.131588, the strategy has improved accuracy
substantially.  

The term {\it correction for continuity} alludes to the fact that we
are approximately a discrete distribution by a continuous one.

\section{Example:  Normal Approximation to Gamma Family}

Recall from above that the gamma distribution, or at least the Erlang,
arises as a sum of independent random variables.  Thus the Central Limit
Theorem implies that the gamma distribution should be approximately
normal for large (integer) values of r.  We see in Figure \ref{gammas}
that even with r = 10 it is rather close to normal.\footnote{It should
be mentioned that technically, the CLT, which concerns convergence of
cdfs, does not imply convergence of densities.  However, under mild
mathematical conditions, convergence of densities occurs too.}

\section{Example:  Museum Demonstration}

Many science museums have the following visual demonstration of the CLT.

There are many balls in a chute, with a triangular array of r rows of pins
beneath the chute.  Each ball falls through the rows of pins, bouncing
left and right with probability 0.5 each, eventually being collected
into one of r bins, numbered 0 to r.  A ball will end up in bin i if it
bounces rightward in i of the r rows of pins, i = 0,1,...,r.  Key point:

\begin{quote}
Let X denote the bin number at which a ball ends up.  X is the number of
rightward bounces (``successes'') in r rows (``trials'').  Therefore X
has a binomial distribution with n = r and p = 0.5
\end{quote}

Each bin is wide enough for only one ball, so the balls in a bin will
stack up.  And since there are many balls, the height of the stack in
bin i will be approximately proportional to P(X = i).  And since the
latter will be approximately given by the CLT, the stacks of balls will
roughly look like the famous bell-shaped curve!

There are many online simulations of this museum demonstration, such as
\url{http://www.mathsisfun.com/data/quincunx.html}.  By collecting the
balls in bins, the apparatus basically simulates a histogram for $X$,
which will then be approximately bell-shaped.

\section{Importance in Modeling}
\label{normalimp}

Needless to say, there are no random variables in the real world that
are exactly normally distributed.  In addition to our comments at the
beginning of this chapter that no real-world random variable has a
continuous distribution, there are no practical applications in which a
random variable is not bounded on both ends.  This contrasts with normal
distributions, which extend from $-\infty$ to $\infty$.

Yet, many things in nature do have approximate normal distributions, so
normal distributions play a key role in statistics.  Most of the
classical statistical procedures assume that one has sampled from a
population having an approximate distribution.  In addition, it will be
seen later than the CLT tells us in many of these cases that the
quantities used for statistical estimation are approximately normal,
even if the data they are calculated from are not.

Recall from above that the gamma distribution, or at least the Erlang,
arises as a sum of independent random variables.  Thus the Central Limit
Theorem implies that the gamma distribution should be approximately
normal for large (integer) values of r.  We see in Figure \ref{gammas}
that even with r = 10 it is rather close to normal.

\section{The Chi-Squared Family of Distributions}
\label{chisqfam}

\subsection{Density and Properties}

Let $Z_1, Z_2,..., Z_k$ be independent N(0,1) random variables.  Then
the distribution of 

\begin{equation}
\label{sumz2}
Y = Z_1^2+...+Z_k^2
\end{equation}

is called {\bf chi-squared with k degrees of freedom}.  We write such a
distribution as $\chi_k^2$.  Chi-squared is a one-parameter family of
distributions, and arises quite frequently in statistical applications,
as will be seen in future chapters.

We can derive the mean of a chi-squared distribution as follows.  First,

\begin{equation}
EY = E(Z_1^2+...+Z_k^2) = k E(Z_1^2)
\end{equation}

Well, $E(Z_1^2)$ sounds somewhat like variance, suggesting that we use
(\ref{varuformula}).  This works:

\begin{equation}
E(Z_1^2) = Var(Z_1) +  [E(Z_1)]^2 = Var(Z_1) = 1
\end{equation}

Then EY in (\ref{sumz2}) is k.  One can also show that Var(Y) = 2k.

It turns out that chi-squared is a special case of the gamma family in
Section \ref{gammafam} below, with r = k/2 and $\lambda = 0.5$.

The R functions {\bf dchisq()}, {\bf pchisq()}, {\bf qchisq()} and {\bf
rchisq()} give us the density, cdf, quantile function and random number
generator for the chi-squared family.  The second argument in each case
is the number of degrees of freedom.  The first argument is the argument
to the corresponding math function in all cases but {\bf rchisq()}, in
which it is the number of random variates to be generated.

For instance, to get the value of $f_X(5.2)$ for a chi-squared random
variable having 3 degrees of freedom, we make the following call:

\begin{lstlisting}
> dchisq(5.2,3)
[1] 0.06756878
\end{lstlisting}

\subsection{Example:  Error in Pin Placement}

Consider a machine that places a pin in the middle of a flat,
disk-shaped object. The placement is subject to error. Let $X$ and $Y$ be
the placement errors in the horizontal and vertical directions,
respectively, and let $W$ denote the distance from the true center to the
pin placement. Suppose $X$ and $Y$ are independent and have normal
distributions with mean 0 and variance 0.04. Let's find $P(W > 0.6)$. 

Since a distance is the square root of a sum of squares, this sounds
like the chi-squared distribution might be relevant.  So, let's first
convert the problem to one involving squared distance:

\begin{equation}
P(W > 0.6) = P(W^2 > 0.36)
\end{equation}

But $W^2 = X^2 + Y^2$, so

\begin{equation}
P(W > 0.6) = P(X^2 + Y^2 > 0.36)
\end{equation}

This is not quite chi-squared, as that distribution involves the sum of
squares of independent N(0,1) random variables.  But due to the normal
family's closure under affine transformations (page \pageref{affine}),
we know that X/0.2 and Y/0.2 do have N(0,1) distributions.  So write

\begin{equation}
P(W > 0.6) = P[(X/0.2)^2 + (Y/0.2)^2 > 0.36/0.2^2]
\end{equation}

Now evaluate the right-hand side:

\begin{lstlisting}
> 1 - pchisq(0.36/0.04,2)
[1] 0.01110900
\end{lstlisting}

\subsection{Example:  Generating Normal Random Numbers}

How do normal random number generators such as {\bf rnorm()} work?
While in principle Section \ref{genrannumgen} could be used, the lack of
a closed-form expression for $\Phi^{-1}()$ makes that approach
infeasible.  Instead, we can exploit the relation between the normal
family and exponential distribution, as follows.

Let $Z_1$ and $Z_2$ be two independent N(0,1) random variables, and define
$W = Z_1^2 + Z_2^2$.  By definition, $W$ has a chi-squared distribution with
2 degrees of freedom, and from Section \ref{chisqfam}, we know that $W$
has a gamma distribution with $r = 1$ and $\lambda = 0.5$.

In turn, we know from Section \ref{gammafam} that that distribution is
actually just an exponential distribution with the same value of
$\lambda$.  This is quite fortuitous, since Section \ref{genrannumgen}
{\it can} be used in this case.

Specifically, $f_W(t) = 0.5 e^{-0.5t}$, so $F_W(t) = 1 - e^{-0.5t}$,
Thus

\begin{equation}
F_W^{-1}(s) = -2 \ln(1-s)
\end{equation}

And there's more.  Think of plotting the pair $(Z_1,Z_2)$ in the X-Y
plane, and define

\begin{equation}
\theta = \tan^{-1}(\frac{Z_2}{Z_1})
\end{equation}

One can show that $\theta$ has a uniform distribution on
$(0,2\pi)$.  Also, 

\begin{equation}
Z_1 = W\cos(\theta), ~~~ Z_1 = W\sin(\theta)
\end{equation}

Putting all this together, we can generate a pair of independent N(0,1)
random variates via the code

\begin{lstlisting}
function genn01() {
   theta <- runif(1,0,2*pi)
   w <- -log(1 - runif(1))
   c(w*cos(theta),w*sin(theta))
}
\end{lstlisting}

Note by the way that we ``get two for the price of one.''  If we need,
say, 1000 random normal variates, we call the above function 500 times.
If we need just variate, we use the first in the returned pair, and save
the second in case we need it later in another call.

\subsection{Importance in Modeling}

This distribution family does not come up directly in application nearly
so often as, say, the binomial or normal distribution family.

But the chi-squared family is used quite widely in statistical
applications.  As will be seen in our chapters on statistics, many
statistical methods involve a sum of squared normal random
variables.\footnote{The motivation for the term {\it degrees of freedom}
will be explained in those chapters too.}

\subsection{Relation to Gamma Family}

One can show that the chi-square distribution with d degrees of freedom
is a gamma distribution, with r = d/2 and $\lambda = 0.5$.

\section{The Multivariate Normal Family}

(Here we borrow some material from Chapter \ref{mul}.)

The generalization of the normal family is the multivariate normal.
Instead of being parameterized by a scalar mean and a scalar variance,
the multivariate normal family has as its parameters a vector mean and
a covariance matrix.

Let's look at the bivariate case first.  The joint distribution of
$X_1$ and $X_2$ is said to be {\bf bivariate normal} if their density is

\begin{equation}
f_{X,Y}(s,t) = \frac{1}{2\pi \sigma_1 \sigma_2 \sqrt{1-\rho^2}}
e^
{-\frac{1}{2(1-\rho^2)} 
\left [
\frac{(s-\mu_1)^2}{\sigma_1^2} + \frac{(t-\mu_2)^2}{\sigma_2^2} 
-\frac
{2\rho (s-\mu_1)(t-\mu2)}
{\sigma_1 \sigma_2}
\right ]
}, ~ -\infty < s,t < \infty
\end{equation}

{\bf This looks horrible, and it is.  But don't worry, as we won't work
with this directly.  It's important for conceptual reasons, as follows.}

First, note the parameters here:  $\mu_1$, $\mu_2$, $\sigma_1$ and
$\sigma_2$ are the means and standard deviations of X and Y,
while $\rho$ is the correlation between X and Y.  So, we have a
five-parameter family of distributions.  The graph of the bivariate
normal density looks like a 3-dimensional bell, as seen on the cover of
this book.

The multivariate normal family of distributions is parameterized by one
vector-valued quantity, the mean $\mu$, and one matrix-valued quantity,
the covariance matrix $\Sigma$.  Specifically, suppose the random vector
$X = (X_1,...,X_k)'$ has a k-variate normal distribution.

The density has this form:

\begin{equation}
f_X(t) = c e^{-0.5 (t-\mu)'\Sigma^{-1}(t-\mu)}
\end{equation}

Here c is a constant, needed to make the density integrate to 1.0. 

There is a Multivariate Central Limit Theorem, that says that sums of
random vectors have approximately multivariate normal distributions.

In R the density, cdf and quantiles of the multivariate normal
distribution are given by the functions {\bf dmvnorm()}, {\bf pmvnorm()} 
and {\bf qmvnorm()} in the library {\bf mvtnorm}.  You can simulate a
multivariate normal distribution by using {\bf mvrnorm()} in the library
{\bf MASS}.

\section{Optional Topic:  Precise Statement of the CLT}
\label{formalclt}

The statement of Theorem \ref{impreciseclt} is not mathematically precise.
We will fix it here, not just for mathematical niceness, but also
because it leads to something called the {\it delta method}, a very
practical tool in statistics.

\subsection{Convergence in Distribution, and the Precisely-Stated CLT}

\begin{definition}
A sequence of random variables $L_1, L_2, L_3,...$ {\bf converges in
distribution} to a random variable $M$ if

\begin{equation}
\lim_{n \rightarrow \infty} P(L_n \leq t) = P(M \leq t), \textrm{ for
all t}
\end{equation}
\end{definition}

Note by the way, that these random variables need not be defined on the
same probability space.

The formal statement of the CLT is:

\begin{theorem}
Suppose $X_1, X_2, ...$ are independent random variables, all having the
same distribution which has mean m and variance $v^2$.  Then

\begin{equation}
\label{cltquant}
Z = 
\frac
{X_1+...+X_n -n m}
{v \sqrt{n}}
\end{equation}

converges in distribution to a N(0,1) random variable.  

\end{theorem}

